\chapter{Modell}
%\section{Hameltonien}
Um den Inversen Farady-Effekt mit Hilfe der Floquet-Theorie in
einem Bandisolator genauer zu untersuchen soll hier ein einfaches 2D-Modell
eines Festkörpers mit vier Positionen (herangezogen)$/$verwendet werden.
Der Bandisolator wird durch
genau wechselnde lokale Energien dargestellt (ADD).
In der Abbildung \ref{fig:sytem}
ist das zu untersuchende System noch einmal bildlich dargestellt.
%
% \begin{figure}
%   \centering
%   \includefigure{}
%   \caption{Modell.}
%   \label{.}
% \end{figure}
\section{Zeitunabhängiges System}
Solch ein System lässt sich durch
ein Tight-Binding-Modell in zweiter Quantisierung beschreiben.
Aus dem Tight-Binding-Modell ergibt sich der folgende Hamiltonian für das zeitunabhängige Systems % lautet
\begin{align}
  H=J\sum_{i=1}^4 \left(c_{i+1}^\dag c_i^{\phantom{\dag}} + c_{i}^\dag c_{i+1}^{\phantom{\dag}}   +a_i^{\phantom{\dag}} c_i^\dag c_i^{\phantom{\dag}}\right).
\end{align}
Durch die Wahl einer Basis, hier
\begin{align}
 \ket{1}=\vec{e_1}&  &\ket{2}=\vec{e_2}&  &\ket{2}=\vec{e_2}& &\ket{3}=\vec{e_3}& &\ket{4}=\vec{e_4},
\intertext{kann über}
H_ji=\braket{j|H|i}
\intertext{die Matrixdarstellung des Hamiltonian}
  H=\begin{pmatrix}
  -a& J& & J \\
  J& a &J & 0\\
  0& J& -a& J \\
  J& 0&  J & a
\end{pmatrix}
\end{align}
bestimmt werden.
Durch die stationäre Schrödinger Gleichung
können nun die Eigenwerte $E_i$ und Eigenvektoren $\ket{\phi_i}$  des Hamiltonian bestimmt werden.
Die Eigenwerte des zeitunabhängigen Systems sind:
\begin{align}
  E_1&=-\sqrt{a^2+4J^2}&  &E_2=-a\\
  E_2&=a& &E_4=\sqrt{a^2+4J^2}.
\end{align}
So ergibt sich das in der Abbildung \ref{fig:bandstrucktur} dargestellte Niveauschema.
%Ebenfalls kann die für ein Bandisolator charakteristische Bandlücke von $2a$
Aus dem Niveauschema können die Resonanzfrequenzen des Systems im Grundzustand entnommen werden.
(Abbildung)

\section{Zeitabhängiges System}
Nun soll das System durch ein sich in der Mitte befindendes rotierendes E-Feld erweitert werden, welches
das zirkular polarisierte Licht simulieren soll (Abbildung:\ref{fig:syst+E}).
Der Hamiltonian wird dafür um den Term
\begin{align}
  \sum_{i=1}^4 \epsilon_i^{\phantom{\dag}} c_i^\dag c_i^{\phantom{\dag}}  \,\,   \text{mit} \,\, \epsilon=-\vec{E} \vec{r_i}
\end{align}
erweitert. Dabei beschreibt $\vec E$ das E-Feld
\begin{align}
  \vec E=E_0\begin{pmatrix}
\cos\left(\omega t\right)\\
\sin\left(\omega t\right)
 \end{pmatrix},
 \intertext{welches die Amplitude $E_0$ besitzt und mit der Frequenz $\omega$ rotiert.}
\end{align}
Der Ursprung des Systems wird einfachshalber in dem Mittelpunkt des Systems gesetzt
somit lässt sich $r_i$, der Vektor welcher (ADD) Position zeigt, in Abhänigigkeit von der Gitterkonstante $d$ schreiben.
Der gesamte Hamiltonian des Systems nimmt somit die Form
\begin{align}
H=J\sum_{i=1}^4 \left(c_{i+1}^\dag c_i^{\phantom{\dag}} + c_{i}^\dag c_{i+1}^{\phantom{\dag}}   +a_i^{\phantom{\dag}} c_i^\dag c_i^{\phantom{\dag}} +\epsilon_i^{\phantom{\dag}} c_i^\dag c_i^{\phantom{\dag}}\right)\\
H=J\sum_{i=1}^4 \left(c_{i+1}^\dag c_i^{\phantom{\dag}} + c_{i}^\dag c_{i+1}^{\phantom{\dag}}   +a_i^{\phantom{\dag}} c_i^\dag c_i^{\phantom{\dag}} -\vec{E} \vec{r_i}  c_i^\dag c_i^{\phantom{\dag}}\right)\\
\end{align}
an.


\section{Floquetmatrix des gestörten Systems}



\begin{align}
H=\underbrace{J\sum_{i=1}^4 \left(c_{i+1}^\dag c_i + c_{i}^\dag c_{i+1}c_i\right)   +\sum_{i=1}^4a_i c_i^\dag c_i}{H_0} +\underbrace{\sum_{i=1}^4\epsilon_i c_i^\dag}{V(t)}
\end{align}

Durch Anwendung von \eqref{eqn:H_n_m} ergibt sich:
\begin{align}
  H^{m-n}=H_0\delta_{m,n} -\frac{E}{2}\left(R_{-} \exp\left( \mathrm{i}\upphi\right)\delta_{n+1,m}   +  R_{+}\exp\left( -\mathrm{i}\upphi\right)\delta_{n-1,m}\right)\\
  R_{-}=
\begin{pmatrix}
  r_{1_x}-r_{1_y}& & & \\
  &r_{2_x}-r_{2_y} & & \\
  & & r_{3_x}-r_{3_y}& \\
  & & & r_{4_x}-r_{4_y}
\end{pmatrix}
\end{align}

\section{Abschätzen der Größenordnungen}
