\chapter{Modell}
%\section{Hameltonien}
Um den Inversen Farady-Effekt mit Hilfe der Floquet Theorie in
einem Bandisolator genauer zu untersuchen, soll hier ein simples 2D-Modell
eines Festkörpers mit vier Gitterplätzen, die eine lokale Energie $a$ besitzen, herangezogen werden.
Um die Eigenschaften des Bandisolators zu simulieren, alterniert
das Vorzeichen der lokalen Energie $a$.
Die Gitterplätze sind im Abstand der Gitterkonstante $d$ quadratisch angeordnet.
In dem System soll sich zunächst nur ein Elektron befinden, welches mit der Energie
$J$ zu den nächsten Nachbarn "hüpfen" kann.
In der Abbildung \ref{fig:system}
ist das zu untersuchende System noch einmal bildlich dargestellt.

\begin{figure}
   \centering
   \includegraphics[width=0.5\textwidth]{C:/Users/daghe/Desktop/Uni/Bachelorarbeit/Tikz_test/bild_gitter_0.pdf}
   \caption{Ein simples 2D-Modell eines Bandisolators mit vier Gitterplätzen im Abstand $d$ und der lokalen Energie $a$}
   \label{fig:system}
\end{figure}




\section{Zeitunabhängiges System}
Solch ein System lässt sich durch
ein Tight-Binding-Modell in zweiter Quantisierung beschreiben.
Aus dem Tight-Binding-Modell ergibt sich der folgende Hamiltonian $H_0$ für das zeitunabhängige System % lautet
\begin{align}
  H_0=J\sum_{i=1}^4 \left(c_{i+1}^\dag c_i^{\phantom{\dag}} + c_{i}^\dag c_{i+1}^{\phantom{\dag}}   +a_i^{\phantom{\dag}} c_i^\dag c_i^{\phantom{\dag}}\right).
\end{align}
Durch die Wahl einer Basis, hier
\begin{align}
 \ket{1}=\vec{e}_{1}&  &\ket{2}=\vec{e}_2&  &\ket{2}=\vec{e}_2& &\ket{3}=\vec{e}_3& &\ket{4}=\vec{e}_4,
\intertext{kann über}
H_{ji}=\braket{j|H|i}
\intertext{die Matrixdarstellung des Hamiltonian}
  H_0=\begin{pmatrix}
  -a          & \phantom{-}J &\phantom{-}0& \phantom{-}J \\
  \phantom{-}J& \phantom{-}a &\phantom{-}J& \phantom{-}0\\
  \phantom{-}0& \phantom{-}J & -a         & \phantom{-}J \\
  \phantom{-}J& \phantom{-}0&\phantom{-}J & \phantom{-}a
\end{pmatrix}
\end{align}
bestimmt werden.
Mit Hilfe der stationären Schrödingergleichung
können nun die Eigenwerte $E_n$ und die Eigenvektoren $\ket{\phi}_n$  des Hamiltonian bestimmt werden.
Es ergeben sich die Eigenwerte des zeitunabhängigen Systems zu:
\begin{align}
  E_{1/4}&=\mp\sqrt{a^2+4J^2}&  &E_{2/3}=\mp a
\end{align}
So folgt das in der Abbildung \ref{fig:bandstrucktur} dargestellte Niveauschema.
\begin{figure}
   \centering
   \includegraphics[width=0.4\textwidth]{C:/Users/daghe/Desktop/Uni/Bachelorarbeit/Tikz_test/bild_niveau.pdf}
   \caption{Niveauschema für ein Elektron im System.}
   \label{fig:bandstrucktur}
\end{figure}

%Ebenfalls kann die für ein Bandisolator charakteristische Bandlücke von $2a$
Aus dem Niveauschema sind die Resonanzfrequenzen des Systems im Grundzustand zu entnehmen:
\begin{align}
\omega_{\text{res}_1}=\sqrt{a^2+4J^2}-a,
& &\omega_{\text{res}_2}=a+\sqrt{a^2+4J^2},
& &\omega_{\text{res}_3}=2\sqrt{a^2+4J^2}.
\end{align}

\section{Zeitabhängiges System}
Nun soll das System durch ein sich in der Mitte befindendes rotierendes E-Feld erweitert werden, welches
das zirkularpolarisierte Licht simulieren soll, zu sehen in der Abbildung \ref{fig:syst+E}.

\begin{figure}
   \centering
   \includegraphics[width=0.4\textwidth]{C:/Users/daghe/Desktop/Uni/Bachelorarbeit/Tikz_test/bild_gitter.pdf}
   \caption{System+E-Feld.}
   \label{fig:syst+E}
\end{figure}


Der Hamiltonian wird dafür um den Term
\begin{align}
  \sum_{i=1}^4 \epsilon_i^{\phantom{\dag}} c_i^\dag c_i^{\phantom{\dag}}  \,\,   \text{mit} \,\, \epsilon=-\vec{E} \vec{r_i}
\end{align}
erweitert. Dabei beschreibt $\vec E$ das E-Feld
\begin{align}
  \vec E=E_0\begin{pmatrix}
\cos\left(\omega t+\upphi\right)\\
\sin\left(\omega t+\upphi\right)
 \end{pmatrix},
\end{align}
welches die Amplitude $E_0$, die Frequenz $\omega$ und die Phase $\upphi$ besitzt.
Der Ursprung des Systems wird einfachheitshalber in den Mittelpunkt des Systems gesetzt,
somit lässt sich der Vektor $\vec{r_i}$, der auf ein Gitterplatz zeigt, in Abhänigigkeit von der Gitterkonstante $d$ schreiben:
\begin{align}
  r_1=\frac{1}{2}\begin{pmatrix}-d  \\ \phantom{-}d \end{pmatrix},& &
  r_2=\frac{1}{2}\begin{pmatrix}-d  \\ -d \end{pmatrix},& &
  r_3=\frac{1}{2}\begin{pmatrix}\phantom{-}d  \\ -d \end{pmatrix},& &
  r_4=\frac{1}{2}\begin{pmatrix}d  \\ d \end{pmatrix}.
\end{align}
Der gesamte Hamiltonian des Systems nimmt somit die Form
\begin{align}
%H=J\sum_{i=1}^4 \left(c_{i+1}^\dag c_i^{\phantom{\dag}} + c_{i}^\dag c_{i+1}^{\phantom{\dag}}   +a_i^{\phantom{\dag}} c_i^\dag c_i^{\phantom{\dag}} +\epsilon_i^{\phantom{\dag}} c_i^\dag c_i^{\phantom{\dag}}\right)\\
H=J\sum_{i=1}^4 \left(c_{i+1}^\dag c_i^{\phantom{\dag}} + c_{i}^\dag c_{i+1}^{\phantom{\dag}}   +a_i^{\phantom{\dag}} c_i^\dag c_i^{\phantom{\dag}} -\vec{E} \vec{r_i}  c_i^\dag c_i^{\phantom{\dag}}\right)
\end{align}
an.


\section{Floquet Matrix des gestörten Systems}
Für das zeitabhängige System soll nun wie
in Kapitel \ref{sec:numerisch} beschrieben, die
numerische Methode der Floquet Matrix angewendet werden.
Dafür kann zunächst der Hamiltonian
\begin{align}
H=\underbrace{J\sum_{i=1}^4 \left(c_{i+1}^\dag c_i^{\phantom{\dag}} + c_{i}^\dag c_{i+1}^{\phantom{\dag}}c_i^{\phantom{\dag}} + a_i^{\phantom{\dag}} c_i^\dag c_i^{\phantom{\dag}}
 \right)}_{H_0} -\underbrace{\sum_{i=1}^4\left(\vec{E} \vec{r_i}  c_i^\dag c_i^{\phantom{\dag}}\right)}_{V(t)}
\end{align}
in den zuvor behandelten zeitunabhänigen Teil $H_0$ und einen in der Zeit periodischen Teil $V(t)$ aufgespalten werden.
Um die Matrix $\mathcal{H}_\mathrm{F}$ zu berechnen, werden zunächst die Untermatrizen $H^{m-n}$ berechnet.
Aus der Gleichung \eqref{eqn:H_n_m} folgt
\begin{align}
  H^{m-n}=H_0\delta_{m,n} -\frac{E_0}{2}\left(R_{(-)} \exp\left( \mathrm{i}\upphi\right)\delta_{n+1,m}   +  R_{(+)}\exp\left( -\mathrm{i}\upphi\right)\delta_{n-1,m}\right)\\
\end{align}
mit
\begin{align}
  R_{(-)}&=\textbf{diag}\left(
  r_{1_x}-r_{1_y} ,
  r_{2_x}-r_{2_y} ,
  r_{3_x}-r_{3_y} ,
  r_{4_x}-r_{4_y}\right)
\\
R_{(+)}&= \textbf{diag}\left(
r_{1_x}+r_{1_y} ,
r_{2_x}+r_{2_y} ,
r_{3_x}+r_{3_y} ,
r_{4_x}+r_{4_y}\right).
\end{align}
Somit ergibt sich die Matrix $\mathcal{H}_F$ nach
der Gleichung \eqref{eqn:H_f} in
Blockschreibweise zu einer Diagonalbandmatrix.
% \begin{align}
%   \mathcal{H}_F=\begin{pmatrix}
%   H^{-n ,-n}+(-n)\hbar\omega  &  H^{-n,-n+1}       &       &    & \\
%   H^{-n+1,-n} &    H^{-n+1,-n+1}+(1-n)\hbar\omega  & \ddots&    & \\
%             &          \ddots                      & \ddots&  \ddots      &    \\
%             &                                      & \ddots&  H^{n-1,n-1}+(n-1)\hbar\omega  & H^{n-1,n}  \\
%             &                                      &       & H^{n,n-1}   & H^{n,n}
% \end{pmatrix}
% \end{align}
Um die exakten $\epsilon_{\alpha}$ und $\ket{\Phi_\alpha}$ zu berechnen, muss die Bedingung
$N\rightarrow\infty $
erfüllt sein.
Dies ist numerisch jedoch nicht möglich, folglich muss die Matrix $\mathcal{H}_F$ bei einem beliebigen $n$ trunkiert werden.
Die Matrix $\mathcal{H}_F$ für eine Trunkierung bei $N=1$ würde beispielsweise diese Form
\begin{align}
  \mathcal{H}_F=\begin{pmatrix}
  H^{-1,-1}-\mathbb{1}\hbar\omega &  H^{-1,0} &   0 \\
  H^{0,-1}               &  H^{0,0}  &H^{0,1}                  \\
      0                  &  H^{1,0}  & H^{1,1}+\mathbb{1}\hbar\omega
\end{pmatrix}
\end{align}
besitzen.
