\chapter{Modellierung des IFE in einem Bandisolator}
\label{sec:model}
%\section{Hameltonien}
Um den IFE mit Hilfe der Floquet Theorie in
einem Bandisolator genauer zu untersuchen, wird hier ein
mikroskopisches zwei dimensionales Modell
eines Festkörpers mit vier Gitterplätzen herangezogen. Zu Beginn
wird ein System betrachtet, in dem sich ein spinloses
Elektron aufhält, wobei zwischen dem
zeitunabhängigen System ohne ein elektrisches Feld
und dem zeitabhängigen System mit einem in der Zeit
periodischen elektrischen Feld unterschieden wird.
Darauffolgend wird die Floquet Matrix nach dem Abschnitt
\ref{sec:matrix} für das
zeitabhängige System aufgestellt. Abschließend wird das
System für zwei spinlose Elektronen betrachtet.


\section{Zeitunabhängiges System}
Dieses Gittersystem wird durch das
Tight-Binding-Modell in zweiter Quantisierung
mit periodischen Randbedingungen \cite{czycholl} beschrieben.
Aus diesem ergibt sich der folgende Hamiltonian $H_0$
für das zeitunabhängige System % lautet
\begin{align}
  H_0=\underbrace{J\sum_{i=1}^4 \left(c_{i+1}^\dag c_i^{\phantom{\dag}} + c_{i}^\dag c_{i+1}^{\phantom{\dag}}\right)}_{H_{TB}}
   +\underbrace{\sum_{i=1}^4\left( a_i^{\phantom{\dag}} c_i^\dag c_i^{\phantom{\dag}} \right)}_{H_a},
\end{align}
wobei der Tight-Binding Hamiltonian $H_{TB}$, der die Sprünge des Elektrons
zwischen den Gitterplätzen beschreibt, mit dem Hamiltonian $H_a$, der die Eigenschaften
eines Bandisolators simuliert, erweitert wird.
Dabei besitzen die Gitterplätze eine lokale alternierende Energie $a$.
Die Gitterplätze sind im Abstand der Gitterkonstanten $d$ quadratisch angeordnet.
In der Abbildung \ref{fig:system}
ist eine Skizze des zu untersuchenden Systems dargestellt.
\begin{figure}
   \centering
   \includegraphics[width=0.4\textwidth]{Programme/Tikz_test/bild_gitter_0.pdf}
   \caption{Skizze des verwendeten 2D-Modells
    des Bandisolators mit vier Gitterplätzen im Abstand $d$,
   welche die alternierende lokale Energie $a$ besitzen.
    Elektronen im System können abhängig von dem Tight-Binding-Paramter $J$
   den Gitterplatz wechseln.}
   \label{fig:system}
\end{figure}
Durch die Wahl der Basis
\begin{align}
\ket{1}=\vec{e}_{1}&  \ket{2}=\vec{e}_2&   &\ket{3}=\vec{e}_3& &\ket{4}=\vec{e}_4,&
\intertext{mit den Einheitsvektoren $\vec{e}_i$ wird über}
H_{ij}&=\braket{i|H|j}
\end{align}
eine Matrixdarstellung des Hamiltonians
\begin{align}
  H_0&=\begin{pmatrix}
  -a          & \phantom{-}J &\phantom{-}0& \phantom{-}J \\
  \phantom{-}J& \phantom{-}a &\phantom{-}J& \phantom{-}0\\
  \phantom{-}0& \phantom{-}J & -a         & \phantom{-}J \\
  \phantom{-}J& \phantom{-}0&\phantom{-}J & \phantom{-}a
\end{pmatrix}
\end{align}
bestimmt.
Mit Hilfe der stationären Schrödingergleichung \ref{eqn:stat}
ergeben sich die Eigenwerte
\begin{align}
  E_{1/4}&=\mp\sqrt{a^2+4J^2}&  &E_{2/3}=\mp a
\end{align}
und die Eigenvektoren $\ket{\phi}_n$  des Hamiltonians.
Die Eigenwerte $E_n$ des Hamiltonians
sind in der Abbildung \ref{fig:bandstruktur}
in einem Niveauschema dargestellt.
\begin{figure}
   \centering
   \includegraphics[width=0.4\textwidth]{Programme/Tikz_test/bild_niveau.pdf}
   \caption{Niveauschema für ein Elektron, welches sich in dem System befindet
und die dadurch resultierende Bandlücke $2a$ und
mögliche Resonanzfrequenzen des Systems.}
   \label{fig:bandstruktur}
\end{figure}

%Ebenfalls kann die für ein Bandisolator charakteristische Bandlücke von $2a$
Die möglichen Resonanzfrequenzen des Systems im Grundzustand
\begin{align}
\omega_{\text{res}_1}=\sqrt{a^2+4J^2}-a,
& &\omega_{\text{res}_2}=a+\sqrt{a^2+4J^2},
& &\omega_{\text{res}_3}=2\sqrt{a^2+4J^2} \label{eqn:Resonanz}
\end{align}
werden aus dem Niveauschema \ref{fig:bandstruktur}
entnommen. In der Abbildung
\ref{fig:bandstruktur} ist ebenfalls die Bandlücke $2a$
dargestellt.

\section{Zeitabhängiges System}
Als nächstes wird das System durch ein rotierendes E-Feld erweitert, welches
das zirkular polarisierte Licht simulieren soll,
zu sehen in der Abbildung \ref{fig:syst+E}.

\begin{figure}
   \centering
   \includegraphics[width=0.4\textwidth]{Programme/Tikz_test/bild_gitter.pdf}
   \caption{Skizze des zeitabhängigen 2D-Modells
bei dem zusätzlich zum zeitunabhängigen Modell
ein mit der Zeit rotierendes elektrisches Feld
eingezeichnet ist.}
 \label{fig:syst+E}
\end{figure}


Der Hamiltonian wird dafür um den Term
\begin{align}
  H_\text{E-Feld}=\sum_{i=1}^4\left(\epsilon_i^{\phantom{\dag}} c_i^\dag c_i^{\phantom{\dag}}\right)& &\text{mit}& &\epsilon&=-\vec{E}(t) \vec{r_i}
\end{align}
erweitert. Dabei beschreibt $H_\text{E-Feld}$ die Welchelwirkung zwischen dem elektrischen Feld
\begin{align}
  \vec E(t)=E_0\begin{pmatrix}
\cos\left(\omega t\right)\\
\sin\left(\omega t\right)
 \end{pmatrix},
\end{align}
welches die Amplitude $E_0$ und
die Frequenz $\omega$ besitzt.
Der Ursprung des Systems wird
in den Mittelpunkt des Systems gesetzt.
Somit lauten die Gitterplatz-Vektoren
$\vec{r_i}$ in Abhänigigkeit von der
Gitterkonstante $d$:
\begin{align}
  \vec{r}_1=\frac{d}{2}\begin{pmatrix}-1  \\ \phantom{-}1 \end{pmatrix},& &
  \vec{r}_2=\frac{d}{2}\begin{pmatrix}-1  \\ -1 \end{pmatrix},& &
  \vec{r}_3=\frac{d}{2}\begin{pmatrix}\phantom{-}1  \\ -1 \end{pmatrix},& &
  \vec{r}_4=\frac{d}{2}\begin{pmatrix}1  \\ 1 \end{pmatrix}.
\end{align}
Der gesamte Hamiltonian des Systems nimmt somit die Form
\begin{align}
%H=J\sum_{i=1}^4 \left(c_{i+1}^\dag c_i^{\phantom{\dag}} + c_{i}^\dag c_{i+1}^{\phantom{\dag}}   +a_i^{\phantom{\dag}} c_i^\dag c_i^{\phantom{\dag}} +\epsilon_i^{\phantom{\dag}} c_i^\dag c_i^{\phantom{\dag}}\right)\\
H=\underbrace{J\sum_{i=1}^4 \left(c_{i+1}^\dag c_i^{\phantom{\dag}} + c_{i}^\dag c_{i+1}^{\phantom{\dag}}\right)}_{H_{TB}}
+\underbrace{\sum_{i=1}^4 \left(a_i^{\phantom{\dag}} c_i^\dag c_i^{\phantom{\dag}}\right)}_{H_a}
+\underbrace{\sum_{i=1}^4 \left(\epsilon^{\phantom{\dag}}_i c_i^\dag c_i^{\phantom{\dag}}\right)}_{H_\text{E-Feld}}
\end{align}
an.


\section{Floquet Matrix des zeitabhängigen Systems}
Für das zeitabhängige System wird wie
in Kapitel \ref{sec:matrix} beschrieben, die
numerische Methode der Floquet Matrix angewendet.
Dafür wird zunächst der Hamiltonian
\begin{align}
H=\underbrace{J\sum_{i=1}^4 \left(c_{i+1}^\dag c_i^{\phantom{\dag}} + c_{i}^\dag c_{i+1}^{\phantom{\dag}}c_i^{\phantom{\dag}}\right)
+ \sum_{i=1}^4\left(a_i^{\phantom{\dag}} c_i^\dag c_i^{\phantom{\dag}}\right)}_{H_0}
-\underbrace{\sum_{i=1}^4\left(\vec{E} \vec{r_i}  c_i^\dag c_i^{\phantom{\dag}}\right)}_{V(t)}
\end{align}
in den zeitunabhängigen Teil $H_0$ und einen in der
 Zeit periodischen Teil $V(t)$ aufgespalten.
Um die Floquet Matrix $\mathcal{H}_\mathrm{F}$ zu berechnen, werden zunächst die Untermatrizen $H^{m-n}$ bestimmt.
Aus der Gleichung \eqref{eqn:H_n_m} folgt
\begin{align}
 H^{m-n}&=H_0\delta_{m,n} -\frac{E_0}{2}\left(R_{(-)} \delta_{n+1,m} + R_{(+)}\delta_{n-1,m}\right)
 \intertext{mit}
  R_{(-)}&=\textbf{diag}\left(
  r_{1_x}-r_{1_y} ;
  r_{2_x}-r_{2_y} ;
  r_{3_x}-r_{3_y} ;
  r_{4_x}-r_{4_y}\right)
\\
  R_{(+)}&= \textbf{diag}\left(
  r_{1_x}+r_{1_y} ;
  r_{2_x}+r_{2_y} ;
  r_{3_x}+r_{3_y} ;
  r_{4_x}+r_{4_y}\right).
\end{align}
Somit ergibt sich die Matrix $\mathcal{H}_F$
nach der Gleichung \eqref{eqn:H_f} in
Blockschreibweise zu einer Diagonalbandmatrix.
% \begin{align}
%   \mathcal{H}_F=\begin{pmatrix}
%   H^{-n ,-n}+(-n)\hbar\omega  &  H^{-n,-n+1}       &       &    & \\
%   H^{-n+1,-n} &    H^{-n+1,-n+1}+(1-n)\hbar\omega  & \ddots&    & \\
%             &          \ddots                      & \ddots&  \ddots      &    \\
%             &                                      & \ddots&  H^{n-1,n-1}+(n-1)\hbar\omega  & H^{n-1,n}  \\
%             &                                      &       & H^{n,n-1}   & H^{n,n}
% \end{pmatrix}
% \end{align}
Für exakte $\epsilon_{\alpha}$ und $\ket{\Phi_\alpha}$ muss
$N\rightarrow\infty$ gehen,
da dies numerisch jedoch nicht möglich ist, wird
die Matrix $\mathcal{H}_F$ bei
einem beliebigen $N$ trunkiert.
Dadurch werden nur die Fouriermoden bis $N$ in der
Berechnung der QEE und QEZ
berücksichtigt.
Die Matrix $\mathcal{H}_F$ besitzt beispielsweise
für einen Trunkierparameter $N=1$ die Form
\begin{align}
  \mathcal{H}_F=\begin{pmatrix}
  H^{-1,-1}-\hbar\omega\mathbb{1} &  H^{-1,0} &   0 \\
  H^{0,-1}               &  H^{0,0}  &H^{0,1}                  \\
      0                  &  H^{1,0}  & H^{1,1}+\hbar\omega\mathbb{1}
\end{pmatrix}.
\end{align}

\section{Zwei-Elektronen-System}
Dem Modell wird ein Elektron
hinzugefügt, in Folge dessen
unterliegt das System dem Pauliverbot.
Die Eigenwerte $E_n$ setzen sich dabei aus
linear Kombinationen der Eigenwerte
des Ein-Elektron-Systems
zusammen \cite{phillip} zu
\begin{align}
E_{1/4}&=\mp\sqrt{a^2+4J^2}-a
&E_{2/6}&=\mp\sqrt{a^2+4J^2}+a
&E_{3/4}&=0.
\end{align}
Somit besitzt das System im Grundzustand die möglichen Resonanzfrequenzen
\begin{align}
\omega_{\text{res}_1}&=2a
&\omega_{\text{res}_{2/3}}&=\sqrt{a^2+4J^2}+a \\
\omega_{\text{res}_4}&=2\sqrt{a^2+4J^2}
&\omega_{\text{res}_5}&=2\sqrt{a^2+4J^2}+2a.
\end{align}
Wird dem Modell wieder ein Elektron
hinzugefügt und es somit auf ein
Drei-Elektronen-System erweitert, ist es
möglich, das Modell als ein System
zu betrachten, welches nur ein
Teilchen mit positiver Ladung enthält.
Diese Betrachtung liefert dasselbe Ergebnis
wie das Ein-Elektron-System und
ist somit trivial. \cite{phillip}
