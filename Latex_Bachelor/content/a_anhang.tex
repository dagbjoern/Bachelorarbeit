\chapter{Ein Anhangskapitel}
\begin{figure}
  \centering
  \begin{subfigure}{0.48\textwidth}
    \includegraphics[width=1\textwidth]{Programme/Energien_kontinuierlich/Plots/Plot_fur_a=1.0_w=1.0N=1.0.pdf}
    \caption{Trunkierparameter $N=1$.}
    \label{fig=N_1}
  \end{subfigure}
  \begin{subfigure}{0.48\textwidth}
    \includegraphics[width=1\textwidth]{Programme/Energien_kontinuierlich/Plots/Plot_fur_a=1.0_w=1.0N=3.0.pdf}
    \caption{Trunkierparameter $N=3.$}
    \label{fig=N_3}
  \end{subfigure}
  \begin{subfigure}{0.48\textwidth}
    \includegraphics[width=1\textwidth]{Programme/Energien_kontinuierlich/Plots/Plot_fur_a=1.0_w=1.0N=5.0.pdf}
    \caption{Trunkierparameter $N=5$.}
    \label{fig=N_5}
  \end{subfigure}
  \begin{subfigure}{0.48\textwidth}
    \includegraphics[width=1\textwidth]{Programme/Energien_kontinuierlich/Plots/Plot_fur_a=1.0_w=1.0N=6.0.pdf}
    \caption{Trunkierparameter $N=6$.}
    \label{fig=N_6}
  \end{subfigure}
  \caption{Berechnete $\epsilon_{\alpha n}$ für unterschiedliche
  Trunkierparameter $N\in\{1,3,5,6\}$
  der Matrix $\mathcal{H}_F$ in Abhängigkeit von der Ampiltude $E_0$
  bei einer lokalen Energie $a=1\,J$ und einer Frequenz
  $\omega=1\,\frac{J}{\hbar}$. Die gestrichelten Linien grenzen dabei
  die verschiedenen Brillouin-Zonen ab.}
  \label{fig:brillouin}
\end{figure}





\begin{figure}
   \centering
   \begin{subfigure}{0.48\textwidth}
       \includegraphics[width=1\textwidth]{Programme/Orthogonalitat_der_quasizustande_frequenz/Plots/Potential=1.0/Energie=0.028/Anzahl=3/Plot_fur_phi1_phi_i.pdf}
       \caption{Trunkierparameter $N=3$}
       \label{fig:N_3}
     \end{subfigure}
     \begin{subfigure}{0.48\textwidth}
       \includegraphics[width=1\textwidth]{Programme/Orthogonalitat_der_quasizustande_frequenz/Plots/Potential=1.0/Energie=0.028/Anzahl=10/Plot_fur_phi1_phi_i.pdf}
       \caption{Trunkierparameter $N=10$}
       \label{fig:N_10}
     \end{subfigure}
     \begin{subfigure}{0.48\textwidth}
       \includegraphics[width=1\textwidth]{Programme/Orthogonalitat_der_quasizustande_frequenz/Plots/Potential=1.0/Energie=0.028/Anzahl=20/Plot_fur_phi1_phi_i.pdf}
       \caption{Trunkierparameter $N=20$}
       \label{fig:N_20}
     \end{subfigure}
     \begin{subfigure}{0.48\textwidth}
       \includegraphics[width=1\textwidth]{Programme/Orthogonalitat_der_quasizustande_frequenz/Plots/Potential=1.0/Energie=0.028/Anzahl=50/Plot_fur_phi1_phi_i.pdf}
       \caption{Trunkierparameter $N=50$}
       \label{fig:N_50}
     \end{subfigure}
     \caption{Das Skalarprodukt $\braket{\Phi_1|\Phi_\alpha}$
      für unterschiedliche Trunkierparameter $N\in\{3,10,20,50\}$
      der Matrix $\mathcal{H}_F$
      in Abhängigkeit von der Frequenz $\omega$
      für $a=1\, J$ und $E_0=\num{0,03}\,\frac{J}{d\symup{e}}$}
    \label{fig:N_gross}
\end{figure}
