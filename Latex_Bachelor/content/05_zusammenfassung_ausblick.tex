\chapter{Zusammenfassung und Ausblick}
\label{sec:zusamm}
vorgehen und Ergebnisse zusammenfassen und Ausblick geben
n dem der
die Ergebnisse aus dem Kapitel \ref{sec:ergbnisse}
zusammengetragen und ein Ausblick auf offene Fragen
gegeben.
Es wurde ein Modell eines Bandisolator aufgestellt welches
mit Hilfe der Floquet Theorie untersucht wurde.
Zunächst wurden Eigenschaften der Floquet Theorie untersucht
dabei wurde die Darstellung in der 1. Brilloun zone für
QEE bestätigt ebenfalls
wurde ein proportionalität  $E_0=\propto N$  zwischen
der Amplitude des elektrischen Feldes $E_0$
und dem zuwählenden Trunkierparameter $N$ gefundnen.
Anschließend wurde die orthogonalität der QEZ überprüft dabei
antiproportionale Zusammenhänge
$\frac{1}{\omega}\propto N$ und  $\frac{1}{a}\propto N$ von
lokaler Energie $a$ und Frequenz des Lichtfeldes
 zu dem trunkierparamter
 festgestellt.
Die wurde zeitentwicklung durch floquet bestätigt
sowie die möglichkeit zeitlich gemittelte erwartungswerte
 zu berechnen. Durch die
 Methode zuvor prognostiziert zusammenhänge von
 $\omega$ und $E_0$ überprüft
 Ergebniss $=>$ für ein elektron im system
 konnte bestätigt werden
ebenfalls für zwei elektronen überprüft jedoch
mit floquet kein strom berechnet.
