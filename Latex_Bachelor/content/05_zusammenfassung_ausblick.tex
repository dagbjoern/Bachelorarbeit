\chapter{Zusammenfassung und Ausblick}
\label{sec:zusamm}
Der Schwerpunkt dieser Arbeit liegt in der Floquet Theorie, welche genutzt wird, um den IFE
in einem Bandisolator zu untersuchen.
Dafür wird ein 2D-Modell des Bandisolators,
welcher sich in einem rotierenden elektrischen
Feld befindet, betrachtet
und der Hamiltonian des Systems aufgestellt.
Die Periodizität des Hamiltonians
ermöglicht die Anwendung der Floquet Theorie.
Dabei liefert die numerische Methode
der Floquet Matrix eine Möglichkeit zur Bestimmung der QEE und QEZ
des Systems.

Zu Beginn des Kapitels \ref{sec:ergebnisse} werden
folgende Proportionalitäten
  $E_0=\propto N, \frac{1}{\omega}\propto N$ und $\frac{1}{a}\propto N$
zwischen dem zu wählenden Trunkierparameter $N$ der Floquet Matrix $\mathcal{H}_F$,
und den Größen des Systems gefunden,
um die Bedingungen, die in Kapitel \ref{sec:floquetheo}
an die QEE und QEZ gestellt werden, und zwar die Periodizität der QEE und die Orthogonalität der QEZ zueinander, zu erfüllen.
% ((Zum einen wurden eine Proportionalität $E_0=\propto N$
% zwischen der Amplitude des elektrischen Feldes und dem Parameter $N$ festgestellt
% und zum anderen die Antipropotianlitäten
% $\frac{1}{\omega}\propto N$ und  $\frac{1}{a}\propto N$
% von der Frequenz $\omega$ und der lokalen Energie $a$ zu dem Parameter $N$
% festgestellt.))
Bei korrekter Wahl des Trunkierparameters $N$
zeigt sich in dem Abschnitt \ref{sec:zeit}, dass die Beschreibung der
Entwicklung
eines Zustandes in der Zeit durch den
Floquet Formalismus möglich ist.

Des Weiteren wird der Stromerwartungswert in dem System untersucht,
wobei nur
der zeitlich gemittelte Stromerwartungswert $\bar{\braket{I}}$
für den IFE von Bedeutung ist.
Es wird eine Methode, die
 es speziell in der Floquet Theorie ermöglicht
im zeitlichen Limes gemittelte Größen wie hier den Stromerwartungswert
zu berechnen, vorgestellt und anhand des Systems verifiziert.

In dem Abschnitt \ref{sec:abbhangig} wird der zeitlich gemittelte Stromerwartungswert
des Ein-Elektron-Systems auf
die in Kapitel \ref{sec:inverfaraday} beschriebenen
Abhängigkeiten bei dem IFE in Isolatoren
$\bar{\braket{I}}\propto\omega$ und
$\bar{\braket{I}}\propto E_0$
untersucht.
Die Abhängigkeiten werden für Frequenzen $\omega$ unterhalb der Resonanzfrequenzen des Systems und
für den betrachteten Bereich der elektrischen Amplitude $E_0$ bestätigt.
Abschließend wird das Zwei-Elektronen-System auf diese Abhängigkeiten untersucht.
Jedoch werden diese nicht für das
System bestätigt.

Für das Zwei-Elektronen-System stellt sich heraus,
dass für die zuvor untersuchten
Größen des Ein-Elektron-Systems
kein zeitlich gemittelter
Stromerwartungswert gegeben ist.
Selbst bei möglichen Resonanzfrequenzen werden
keine Resonanzeffekte beobachtet.
Des Weiteren wird bei dem  Zwei-Elektronen-System
für Frequenzen und elektrische Feldamplituden, obwohl diese die Bedingungen an die QEE und QEZ erfüllen, eine Abweichung
des zeitlich gemittelten Stomerwartungswertes festgestellt.
Diese Abweichung ist nicht zu erklären und es stellt sich die Frage, ob einer weitere Bedingung zu
erfüllen ist, um das Zwei-Elektronen-System mit der Floquet Theorie richtig zu beschreiben.

%!!!!!!!!!!!!!!!!!!!!!!

Die beobachteten Unstimmigkeiten zwischen der Theorie und den Ergebnissen
sind mögliche Anzeichen dafür, dass
entweder die
Ableitung der Theorie für den IFE in Isolatoren
über den IFE in Metallen aus dem Abschnitt \ref{sec:inverfaraday}
unzureichend ist, um zutreffende Aussagen zu dem IFE in Isolatoren zu machen, oder
das aufgestellte Modell des Bandisolators in Kapitel \ref{sec:model} nicht ausreicht,
um IFE in einem Bandisolator zu untersuchen.
Mögliche Lösungsansätze sind zum einen die Vergrößerung des Modells und die Berücksichtigung von
Elektronen mit Spin und somit auch der Spinwechselwirkungen unter diesen, wodurch eine
genauere Beschreibung eines Isolators erreicht wird.
Zum anderen ist eine Theorie erforderlich, die quantenmechanische Effekte bei der
Herleitung des IFE nicht vernachlässigt, um mögliche Fehlerquellen zu minimieren.
Ebenfalls wäre eine experimentelle Untersuchung des IFE in einem Bandisolator hilfreich,
um die Ergebnisse zu verifizieren oder zu widerlegen.

% , da möglicherweise quantenmechanische Effekte in
% Isolatoren nicht vernachlässigbar sind.
% !!!!!!Es könnte weiterhin, da in dem Modell nur
% spinlose Elektronen betrachtet wurden,
% ein Modell für einen
% Isolator, welches Spinwechselwirkungen berücksichtigt beispielsweise das Hubbard-Modell \cite{czycholl},
% zur bessern Beschreibung herangezogen werden.!!!!!
%
%
% \chapter{Zusammenfassung und Ausblick}
% \label{sec:zusamm}
% Der Schwerpunkt/Fokus dieser Arbeit liegt auf der Floquet Theorie welche genutzt wird, um den IFE
% in einem Bandisolator zu untersuchen.
% Dafür wurde eine 2D-Modell des Bandisolators,
% wecher sich in einem rotierenden elektrischen
% Feld befindet, betrachtet
% und der Hamiltonian des Systems aufgestellt.
% Die Periodizität des Hamiltonians
% ermöglichte die Anwendung der Floquet Theorie.
% Dabei lieferte die numerische Methode
% der Floquet Matrix eine Möglichkeit zur Bestimmung der QEE und QEZ
% des Systems.
% Zu Beginn des Kapitel \ref{sec:ergebnisse} wurden
% foglende Proportionalitäten
% \begin{align}
%   E_0&=\propto N,&  &\frac{1}{\omega}\propto N,& &\frac{1}{a}\propto N
% \end{align}
% zwischen dem zu wählenden Trunkierparameter $N$ der Floquet Matrix $\mathcal{H}_F$,
% und den Größen des Systems gefunden,
% um die Bedingungen, welche im Kapitel \ref{sec:floquetheo}
% an die QEE und QEZ gestellt werden, zu erfüllen.
% % ((Zum einen wurden eine Proportionalität $E_0=\propto N$
% % zwischen der Amplitude des elektrischen Feldes und dem Parameter $N$ festgestellt
% % und zum anderen die Antipropotianlitäten
% % $\frac{1}{\omega}\propto N$ und  $\frac{1}{a}\propto N$
% % von der Frequenz $\omega$ und der lokalen Energie $a$ zu dem Parameter $N$
% % festgestellt.))
% Bei korrekter Wahl des Trunkierparameters $N$
% wurde in dem Abschnitt \ref{sec:zeit} gezeigt, dass die Beschreibung der
% Entwicklung
% eines Zustandes in der Zeit durch den
% Floquet Formalismus möglich ist.
% Des Weiteren wurde der Strom in dem System untersucht,
% wobei nur
% der zeitlich gemittelte Stromerwartungswert $\bar{\braket{I}}$
% für den IFE von Bedeutung ist.
% Es wurde eine Methode vorgestellt und bestätigt,
% welche es ermöglicht in der Floquet Theorie
% im zeitlichen Limes gemittelte Stromerwartungswerte
% zu berechnen.
% In dem Abschnitt \ref{sec:abbhangig} wurde der gemittelte Stromerwartungswert
% des Ein-Elektron-Systems auf
% die in Kapitel \ref{sec:inverfaraday} beschriebenen
% Abhängigkeiten untersucht und es
% wurden die Proportionalitäten
% $\bar{\braket{I}}\propto\omega$  und
% $\bar{\braket{I}}\propto E_0$
% bestätigt.
% Abschließend wurde das Zwei-Elektronen-System
% ebenfalls auf diese Abhängigkeiten untersucht.
% Jedoch konnten diese nicht für das
% System bestätigt werden.
% Für das Zwei-Elektronen-System stellte sich heraus,
% dass für die zuvor untersuchten
% Größen des Ein-Elektron-Systems
% kein zeitlich gemittelter
% Stromerwartungswert gegeben ist.
% Selbst bei möglichen Resonanzfrequenzen wurden
% keine Resonanzeffekte beobachtet.
% Als mögliche Ursache kommt das Pauli-Prinzip infrage.
% Als Folge dessen
% existieren entweder Fehler in dem
% aufgestellten Modell des Isolators in Kapitel \ref{sec:model}
% oder die Ableitung der Theorie für den IFE in Isolatoren
% über den IFE in Metallen aus dem Abschnitt \ref{sec:inverfaraday}
% ist nicht möglich, da möglicherweise quantenmechanische Effekte in
% Isolatoren nicht vernachlässigbar sind.
% !!!!!!Es könnte weiterhin, da in dem Modell nur
% spinlose Elektronen betrachtet wurden,
% ein Modell für einen
% Isolator, welches Spinwechselwirkungen berücksichtigt beispielsweise das Hubbard-Modell \cite{czycholl},
% zur bessern Beschreibung herangezogen werden.!!!!!
