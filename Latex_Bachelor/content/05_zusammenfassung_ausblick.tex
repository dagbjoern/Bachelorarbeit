\chapter{Zusammenfassung und Ausblick}
\label{sec:zusamm}
Der Fokus dieser Arbeit lag auf der Überprüfung
der Annahme, dass sich der
IFE in einem Bandisolator durch
die Floquet Theorie beschreiben lässt.
Dafür wurde eine 2D Modell des Bandisolators,
wecher sich in einem rotierenden elektrischen
Feld befinded, betrachtet
und  der Hamiltonian des Systems aufgestellt.
Die Periodizität des Hamiltonian ermöglichte die Anwendung der Floquet Theorie.
Dabei lieferte die numerische Methode
der Floquet Matrix eine Möglichkeit zur Bestimmung der QEE und QEZ
des Systems.
Zu Beginn des Kapitel \ref{sec:ergebnisse} wurden
foglende Proportionalitäten
\begin{align}
  E_0&=\propto N,&  &\frac{1}{\omega}\propto N,& &\frac{1}{a}\propto N
\end{align}
zwischen dem zu wählenden Trunkierparameter $N$ der Floquet Matrix $\mathcal{H}_F$,
und den Größen des Systems gefunden,
um die Bedingungen, welche im Kapitel \ref{sec:floquetheo}
an die QEE und QEZ gestellt sind, zu erfüllen.
% ((Zum einen wurden eine Proportionalität $E_0=\propto N$
% zwischen der Amplitude des elektrischen Feldes und dem Parameter $N$ festgestellt
% und zum anderen die Antipropotianlitäten
% $\frac{1}{\omega}\propto N$ und  $\frac{1}{a}\propto N$
% von der Frequenz $\omega$ und der lokalen Energie $a$ zu dem Parameter $N$
% festgestellt.))
Bei korrekte Wahl des Trunkierparameter $N$
wurde in dem Abschnitt \ref{sec:zeit} gezeigt, dass die Beschreibung der
Entwicklung
eines Zustandes in der Zeit durch den Floquet Formalismus möglich ist.
Des Weiteren wurde der Strom in dem System untersucht,
wobei nur
der zeitlich gemittelte Stromerwartungswert $\bar{\braket{I}}$
für den IFE von Bedeutung ist.
Es wurde eine Methode vorgestellt und bestätigt,
welche es ermöglich in der Floquet Theorie
im zeitlichen Limes gemittelte Stromerwartungswerte
zu berechnen.
In dem Abschnitt \ref{sec:abbhangig} wurde der gemittelte Stromerwartungswert
des Ein-Elektron-Systems auf
die in Kapitel \ref{sec:inverfaraday} beschriebenen
Abhängigkeiten untersucht und es
wurden die Proportionalitäten
$\bar{\braket{I}}\propto\omega$  und
$\bar{\braket{I}}\propto E_0$
bestätigt.
Abschließend wurde das Zwei-Elektronen-System
ebenfalls auf diese Abhängigkeiten untersucht.
Jedoch konnten diese nicht für das
System betätigt werden.
Für das Zwei-Elektronen System konnten
ebenfalls gezeigt werden,
dass für die zuvor verwendeten
Größen des ein-elektron-system
ein so geringer
 stromfluss in dem zwei-elektronen-system
existiert welcher als null interpretiert werden kann.
selbst bei möglichen resonanzfrequnzen wurden
bei keine Resonanzeffekte beobachtet.
als mögliche Ursache kommt das pauliverbot infrage.
somit existieren entweder Fehler in dem
aufgestellten Modell
oder in der Theorie für den IFE.
es wurden nur spinlose Elektronen betrachtet
mögliche veränderung bei
betachtung von elektronen mit Spin
=> hubbard Modell
Möglcihe fehler in der theorie das IFE
da versucht wurde von dem IFE in metallen
aussagen zu dem IFE in isolatoren
zu machen.
=> gesonderte theorie von IFE in isolatoren
Nötig für volständige Berschreibung
