\chapter{Ergebnisse}
\label{sec:ergebnisse}
In diesem Kapitel werden zu Beginn die im Abschnitt  \ref{sec:floquetheo} vorgestellten
Eigenschaften der Floquet Theorie an dem Modell überprüft.
Des weiteren wird die zeitliche Entwicklung
eines Zustandes, welcher durch den
Floquet Formalismus berechnet wird, mit der einer
numerischen Lösung der Schrödinger Gleichung verglichen.
Es wird eine Methode vorgestellt mit der es in der
Floquet Theorie möglich ist zeitlich gemittelte Ströme zu berechnen.
Diese wird durch das Heranziehen des
auf konventionelle Art berechneten zeitlich gemittelten Strom überprüft.
Anschließend wird der berechnete Strom auf in dem Abschnitt
\ref{sec:} prognostizierten Abhänigkeiten untersucht.
Abschließend wird das Zwei-Elektronen-System betrachtet und
ebenfalls ein gemittelter Strom berechnet.

In den folgenden Rechnungen wird der Koeffizient des Tight-Binding
Hamitonien $J$ auf $1\si{\electronvolt}$ gesetzt und
für den Gitterabstand wird eine typischen Wert von $d=\SI{4}{\angstrom}$ gewählt.
Alle anderen Größen werden im Folgenden in Einheiten von $J$ und $d$ angegeben.
Die Bandlücke $2a$ sollte in den Größenordnung des Sprungtermes $J$ liegen und wird
daher auf $2\,J$ gesetzt.
Die Parameter des elektrischen Feldes werden
im Bezug auf das in \cite{jackl} beschriebene Experiment gewählt.
Die Amplitude der Elektrischenfeldes \cite{phillip} liegt somit um den Wert
\begin{align}
  E_0\approx\SI{69,2}{\mega\volt\per\meter}\approx\num{0,03}\, \frac{J}{d\symup{e}}.
\end{align}
Eine typische Frequenz des Lichtfeldes mit einer Wellenlänge $\lambda\approx\SI{800}{\nano\meter}$ ist
\begin{align}
  \omega\approx\SI{2,35}{\peta\hertz}\approx\num{1,5}\,\frac{J}{\hbar}.
\end{align}
Des weiteren wird eine Pulsdauer
\begin{align}
 \mathcal{T}=\SI{50}{\femto\second}\approx76\,\frac{\hbar}{J}
\end{align}
für das Lichtfeld eingeführt, indem das System beobachtet wird.
Eigenwerte und Eigenvektoren einer Matrix werden mit der Funktion \textit{eig} aus dem Programm Octave \cite{octave}
berechnet. Weitere Rechnungen wurden mit Pyhon ...
Wichtige Naturkonstanten werden \cite{schwabl} entnommen.



\section{Frequenzabhängigkeit der Quasienergien}
\label{sec:w_abb}
Zunächst werden die Eigenschaften der Floquet Theorie,
teils unabhängig von typischen Größen des Systems, untersucht. Dafür
werden die Eigenwerte
 $\epsilon_{\alpha n}$ für eine
Matrix $\mathcal{H}_\mathrm{F}$ mit Trunkierparameter $N=1$,  konstanter lokaler Energie $a=1\,J$
und elektrische Feldamplitude $E_0=1\,\frac{J}{d\symup{e}}$
für Frequenz in einem Bereich von $\omega\in\left[0,6\right]\,\frac{J}{\hbar}$ berechnet.
Um die Effekte der Störung $V(t)$ zu verdeutlichen,
werden ebenfalls Eigenwerte $E_{\alpha n}$ für Matrix $\mathcal{H}_F$ mit $E_0=0\,\frac{J}{d\symup{e}}$, dies beschreibt den
zeitunabhänigen Fall, berechnet, sodass in Abbildung \ref{fig:epsilon_f}
die berechneten Eigenwerte verglichen werden können.
\begin{figure}
   \centering
   \includegraphics[width=0.7\textwidth]{Programme/Freqenzen_kontinuierlich/Plots/Plot_fur_a=1.0_E=1.0.pdf}
   \caption{Eigenwerte $\epsilon_{\alpha n}$ der Matrix $\mathcal{H}_\mathrm{F}$
    für $a=1\,J$ und $E_0=1\,\frac{J}{d\symup{e}}$ und
   Eigenwerte $E_{\alpha n}$ des zeitunabhängigen System $E_0=0\,\frac{J}{d\symup{e}}$
    für einem Trunkierparameter $N=1$ in Abhängigkeit von der Frequnez $\omega$.}
   \label{fig:epsilon_f}
\end{figure}
In der Abbildung \ref{fig:epsilon_f} können, wie in \cite{haenggi} beschrieben,
an Stellen wo es zu
Überschneidung der Eigenwerte  $E_{\alpha n}$
des zeitunabhängigen Systems kommt, die Phänomene des "avoided crossing" und
"exact crossing" der Quasienergien $\epsilon_{\alpha n}$ beobachtet werden,
die durch das zeitabhängige elektrische Feld hervorgerufen werden.
??vielleicht??


\section{Brillouin-Zone der Quasienergien}
\label{sec:E_abb}
In diesem Abschitt wird die Darstellung/Darstellbarkeit der Quasienergien, wie in
Abschitt \ref{sec:floquetheo}
beschrieben, in der 1.
Brillouin-Zone untersucht.
Dafür werden Quasienergien bei einer konstanten Frequenz $\omega=1\,\frac{J}{\hbar}$ und
lokalen Energie $a=1\,J$ in Abhängigkeit von der
Amplitude $E_0$, die hier zur Veranschaulichung im
Bereich $E_0\in\left[0,5\right]\,\frac{J}{d\symup{e}}$ liegt, berechnet.
Diese sind in der Abbildung \ref{fig:brillouin} gegen $E_0$ aufgetragen.
Zusätzlich wird der Trunkierparameter $N\in\{1,3,5,6\}$ der Matrix $\mathcal{H}_F$ variiert.

\begin{figure}
  \centering
  \begin{subfigure}{0.48\textwidth}
    \includegraphics[width=1\textwidth]{Programme/Energien_kontinuierlich/Plots/Plot_fur_a=1.0_w=1.0N=1.0.pdf}
    \caption{Trunkierparameter $N=1$.}
    \label{fig=N_1}
  \end{subfigure}
  \begin{subfigure}{0.48\textwidth}
    \includegraphics[width=1\textwidth]{Programme/Energien_kontinuierlich/Plots/Plot_fur_a=1.0_w=1.0N=3.0.pdf}
    \caption{Trunkierparameter $N=3.$}
    \label{fig=N_3}
  \end{subfigure}
  \begin{subfigure}{0.48\textwidth}
    \includegraphics[width=1\textwidth]{Programme/Energien_kontinuierlich/Plots/Plot_fur_a=1.0_w=1.0N=5.0.pdf}
    \caption{Trunkierparameter $N=5$.}
    \label{fig=N_5}
  \end{subfigure}
  \begin{subfigure}{0.48\textwidth}
    \includegraphics[width=1\textwidth]{Programme/Energien_kontinuierlich/Plots/Plot_fur_a=1.0_w=1.0N=6.0.pdf}
    \caption{Trunkierparameter $N=6$.}
    \label{fig=N_6}
  \end{subfigure}
  \caption{Berechnete $\epsilon_{\alpha n}$ für unterschiedliche
  Trunkierparameter $N\in\{1,3,5,6\}$
  der Matrix $\mathcal{H}_F$ in Abhänigkeit von der Ampiltude $E_0$
  bei einer lokalen Energie $a=1\,J$ und einer Frequenz
  $\omega=1\,\frac{J}{\hbar}$. Die gestrichelten Linien grenzen dabei
  die verschiedenen Brillouin Zonen ab.}
  \label{fig:brillouin}
\end{figure}

Es wird deutlich, dass für kleine Trunkierparamter $N$ die Bedingung
für eine Darstellung in der 1. Brillouin-Zone nicht erfüllt ist.
Zudem wird beobachtet, dass die Übereinstimmung
der 2. Brillouin-Zonen mit der 1. Brillouin-Zone mit
größeren Parameter $N$ zunimmt.
Des weiteren zeigt sich eine Abhängigkeit zu der Ampiltude $E_0$, für
Größere Ampiltuden wird ebenfalls ein größerer Trunkierparameter $N$
benötigt um die korrekten QEE zu berechnen.
Diese Beobachtungen zeigen, dass es möglich ist die
QEE $\epsilon_\alpha$  für ein ausreichend großen Trunkierparameter
 $N$ in die erste Brillouin-Zone
mit $\epsilon_\alpha \in\left[-\frac{\omega}{2},\frac{\omega}{2}\right]$ darzugestellt.
Alle anderen QEE $\epsilon_{\alpha n}$ lassen sich somit
durch die Periodizitätsbedingung \eqref{eqn:epsilon_n} berechnen.
In der ersten Brillouin-Zone, siehe Abbildung \ref{fig:brillouin},
existieren genau vier QEE, folglich besitzt das System
vier QEE und vier QEZ.
Aus der Abbildung \ref{fig:brillouin} wird entnommen, dass
für Rechnungen mit einer Amplitude $E_0<1\,\frac{J}{d\symup{e}}$ ein
Trunkierparamter $N=3$ genügt, um die korrekten QEE zu berechnen.
%Aus der  Abbildung \ref{fig:brillouin} kann entnommen werden, System vier Quasienergien besitzt

\section{Orthogonalität der Quasizustände}
\label{sec:ortho}
Im Folgenden wird die Orthogonalität zwischen den
QEZ $\ket{\Phi_\alpha}$ aus der Gleichung
\eqref{eqn:ortho} für
unterschiedliche Trunkierparameter $N$ der Matrix $\mathcal{H}_F$ überprüft.
Dafür wird $\braket{\phi_\beta\vert\phi_\alpha}$
in Abhängigkeit von dem Parameter $N$
für die Werte    $a=1\, J$, $E_0=0,03\,\frac{J}{d\symup{e}}$  und $\omega=1\,\frac{J}{\hbar}$ untersucht.
In der Abbilung \ref{fig:ortho} wird nur $\beta=1$ dargestellt, da $\beta\in\{2,3,4\}$ identische Ergebnisse liefert.

\begin{figure}
    \centering     % vielleicht mit 2x2 subfigure
    \includegraphics[width=0.7\textwidth]{Programme/Orthogonalitat_der_quasizustande/Plots/Potential=1.0/Energie=0.03/Frequenz=1.0/Plot_fur_phi1_phi_i}
    \caption{Das Skalarprodukt $\braket{\Phi_1|\Phi_\alpha}$ in Abhängigkeit von dem Trunkierparameter $N$ der Matrix $\mathcal{H}_F$ für
    $a=1\, J$ , $E_0=0,03\,\frac{J}{d\symup{e}}$  und $\omega=1\,\frac{J}{\hbar}$. }
     \label{fig:ortho}
\end{figure}
Aus Abbildung \ref{fig:ortho} wird entnommen, dass bei bei geringem Parameter $N$ die Orthogonalität der Quasizustände
nicht gegeben ist.
Für die in Abbildung \ref{fig:ortho} verwendeten Größen
erweist sich der Wert $N=3$ als ausreichend zur Gewährleistung der Orthogonalität.
Es stellt sich weiterhin ein antiproportionler Zusammenhang von lokaler Energie $a$ und Frequenz $\omega$ zu der
benötigten Parameter $N$ heraus.
Wegen der antiproportionalität ist es notwendig,
bei den folgenden Berechnungen für
Werte von $\omega<1\,\frac{J}{\hbar}$ und $a<1\,J$ , die
Orthogonalität der Quasizustände zu überprüfen.
Andernfalls reicht eine Matrixgröße von $N=3$ zum Erfüllen der Orthogonalitätsbedingung aus.



% -über formel .. kann quasizustand berechnet werden
% -die Orthogonalitäts bedingung formel .. sollte erfüllt sein

\section{Zeitentwicklung eines Zustandes durch den Floquet Formalismus}
\label{sec:zeit}
In diesem Abschnitt wird eine Zeitentwicklung des Ein-Elektron-Systems mit
oszillierendem E-Feld durch den Floquet Formalismus bestimmt.
Als Startzustand $\ket{\Psi_0}$ wird der Grundzustand $\ket{\phi_1}$ des zeitunabhänigen Systems verwendet, da
dass das System zu Beginn im Zustand niedrigster Energie verweilt.
Es soll die Zeitentwickung für eine Frequenzen $\omega$ des E-Feldes abseits einer Resonanzfrequenz des Systemms
betrachtet werden, welche für ein System mit $a=1\,J$ aus der Tabelle \ref{tab:w_res} entnommen werden können.
\begin{table}
  \centering
  \caption{Resonanzfrequenzen $\omega_{\text{res}_i}$ in einem System mit $a=1 J$, welche sich nach Gleichung \eqref{eqn:Resonanz} berechnen.    }
  \label{tab:w_res}
\begin{tabular}{c | c c c}
\toprule
$a/\,J$ & $\omega_{\text{res}_1}/ \frac{J}{\hbar}$ & $\omega_{\text{res}_2}/\frac{J}{\hbar}$ & $\omega_{\text{res}_3}/\frac{J}{\hbar}$ \\
\midrule
\num{1}  &  \num{1,24}  & \num{3,24} &  \num{4,47}  \\
\bottomrule
\end{tabular}
\end{table}

% -oder -
%
% In der Abbildung \ref{fig:Resonanz} sind
% die Resonanzfrequenzen \eqref{eqn:Resonanz} in Abhängigkeit
% von den lokalen Energien aufgetragen.
%
% \begin{figure}
%   \centering
%   \includegraphics[width=0.7\textwidth]{Programme/Eigenzustande/Plots/Resonanzen.pdf}
%   \caption{Resonanzfrequenzen in Abhängigkeit von der lokalen Energie $a$}
%   \label{fig:Resonanz}
% \end{figure}


Unter dem Ausschluss der Resonanzfrequenzen aus der
Tabelle \ref{tab:w_res},
  % /Abbildung \ref{fig:Resonanz},
wird bei einer lokalen Energie von
$a=\num{1}\, J$ und einer Amplitude von $E_0=\num{0,03}\,\frac{J}{d\symup{e}}$
die Frequenz $\omega=\num{1,5}\,\frac{J}{\hbar}$ gewählt.
Für diese Werte wurde
in den Abschnitten \ref{sec:E_abb} und \ref{sec:ortho} gezeigt, dass ein
Trunkierungsparameter von $N=3$ ausreicht, um die QEE und QEZ korrekt zu berechnen.
Um die Zeitentwicklung nach Floquet durchführen zu können, muss
zunächst durch die  Gleichung \eqref{eqn:super} der Grundzustand
als Superposition der QEZ ausgedrückt werden.
Durch Anwenden des Zeitpropagators \eqref{eqn:Propagator} auf den Grundzustand,
ist es somit möglich den zeitentwickelten Grundzustand
$\ket{\Psi(t)}$ zu berechnen.

In Abbildung \ref{fig:zeitentwicklung} ist
die Aufenthaltswahrscheinlichkeit
$P_i(t)=\lvert\braket{i|\Psi(t)}\rvert^2$ für
die vier unterschiedlichen Gitterplätze in Abhängigkeit von der Zeit dargestellt.
Dabei wird ebenfalls die Aufenthaltswahrscheinlichkeit $P_{i_\text{num}}(t)$ einer numerischen Lösung
der Schrödingergleichung für den Grundzustand durch den
adaptiven Algorithmus $\textit{lsode}$,
??der durch das Programm Octave \cite{octave} bereitgestellt wird,?? aufgetragen.

\begin{figure}
  \centering
  \includegraphics[width=0.7\textwidth]{Programme/Eigenzustande/Plots/Potential=1.0/Energie=0.03/Besetzungen(t)_mit_Floquet_N=3w=1.5.pdf}
  \caption{Aufenthaltswahrscheinlichkeit des zeitenwickelen Grundzustandes für den $i$-ten Gitterplatz
  sowohl von dem Floquet Formalismus $P_i$ als auch von eine numerische Lösung $P_{i_\text{num}}$
  in Abhängigkeit von der Zeit $t$ für
  $a=1\,J$, $\omega=\num{1,5}\,\frac{J}{\hbar}$ und  $E_0=\num{0,03}\,\frac{J}{d\symup{e}}$.
  Die gestrichelten Linien makieren dabei die Periodendauer $T$ des elektrischen Feldes und
  der schwarze Balken die zu Beginn erwähnte Pulsdauer $\mathcal{T}$ des Licht-Feldes.}
  \label{fig:zeitentwicklung}
\end{figure}

Es zeigt sich, dass die verschiedenen Lösungswege die selben Ergebnisse liefert.
Somit ist die Richtigkeit der Zeitentwicklung in der Floquet Theorie bestätigt.
Des weiteren ist erkennbar, dass die Gitterplätze $1$ und $3$
bevorzugt sind. Dies ist durch die Struktur des Bandisolators begründet. Die
die beiden Position besitzen
eine geringere lokale Energie $a$ als die anderen beiden Plätze.
%
% -wenn Zustände orthogonal zeitentwicklung möglich
% -vergleich mit lsode
% -stimmt überein eignet sich folglich für zeitentwicklung

\section{Untersuchung des Stromes in einem Bandisolator}
Die in Abschnitt \ref{sec:zeit} bestätigte Zeitentwicklung eines Zustandes durch den Floquet Formalismus
wird in diesem Abschnitt zur Berechnung des Stromflusses im System, welches sich zu Beginn im Grundzustand befindet,  genutzt.
Der Erwartungswert des Stromflusses im System ergibt sich aus dem Erwartungswert
\begin{align}
\braket{I}(t)=\braket{\psi(t)\lvert I \rvert\psi(t)}
\intertext{des Stromdichteoperators\cite{schwabl}}
I= \symup{i}\frac{J}{4}\sum_i^4 c_{i+1}^\dag c_{i}^{\phantom{\dag}}  -c_{i}^\dag c_{i+1}^{\phantom{\dag}}.
\end{align}
Es werden wieder die Einheitsvektoren $\ket{e_i}$ als Basis gewählt, sodass
sich die Matrixdarstellung des Stromdichteoperators
\begin{align}
I=\symup{i}\frac{J}{4}\begin{pmatrix}
  \phantom{-}0&           -1 &\phantom{-}0 & \phantom{-}1 \\
  \phantom{-}1& \phantom{-}0 &          -1 & \phantom{-}0\\
  \phantom{-}0& \phantom{-}1 &\phantom{-}0 &           -1 \\
            -1& \phantom{-}0 &\phantom{-}1 & \phantom{-}0
\end{pmatrix}
\end{align}
ergibt.
Die Abbildung \ref{fig:strom_t} enthält den Erwartungswert des Stromflusses $\braket{I}(t)$ in Abhängigkeit von der Zeit.
\begin{figure}
  \centering
  \includegraphics[width=0.7\textwidth]{Programme/Strom/Plots/Potential=1.0/Energie=0.03/Stromerwartungswert(t)_N=3w=1.5.pdf}
  \caption{Erwartungswert des Stromflusses $\braket{I}(t)$ im System in Abhängigkeit der Zeit $t$ für
  $a=1\,J$, $\omega=\num{1,5}\,\frac{J}{\hbar}$ und  $E_0=\num{0,03}\,\frac{J}{d\symup{e}}$.
  Die gestrichelten Linien makieren dabei die Periodendauer $T$ des elektrischen Feldes und
  der schwarze Balken die zu Beginn erwähnte Pulsdauer $\mathcal{T}$ des Licht-Feldes.}
 \label{fig:strom_t}
\end{figure}

Da für die Magnetisierung $M$, wie in dem Abschnitt \ref{sec:inverfaraday} beschrieben, nur ein in der Zeit gemittelter
Stromfluss beiträgt, wird hier der Erwartungswert $\braket{I}$ des Stromflusses über eine Zeitspanne $T$
gemittelt. Der zeitlich
gemittelte Erwartungswerte des Stromflusses $\bar{\braket{I}}$ ergibt sich aus
\begin{align}
  \bar{\braket{I}}= \frac{1}{T}\int_0^{T} \braket{\Psi(t)|I|\Psi(t)}. \label{eqn:mittelwert}
\end{align}
Mit Hilfe des Zeitpropagators \eqref{eqn:Propagator}
ist es in dem Floquet Formalismus möglich, zeitlich
gemittelte Erwartungswerte von
Operatoren wie dem Stromdichteoperator zu berechnen \cite{haenggi}.
Durch Einsetzen von \eqref{eqn:psi_t} und \eqref{eqn:fourier}
 in die Gleichung \eqref{eqn:mittelwert} ergibt sich nach Ausführung der Integration
 \begin{align}
 \bar{\braket{I}}&= \lim_{N\to\infty}\sum_\alpha \lvert c_\alpha \rvert^2  \sum_{-N}^{N} \braket{c_\alpha^n\lvert I \rvert c_\alpha^n}  \label{eqn:mittel}
 \intertext{mit}
  c_\alpha&=\braket{\Psi_0\vert\Phi_\alpha}.
\end{align}

In dem Floquet Formalismus ist es somit nicht
notwendig erst die Erwartungswerte $\braket{I}(t)$
zu bestimmen und den Mittelwert über die Gleichung \eqref{eqn:mittelwert}
zu berechnen. Durch die Formel \eqref{eqn:mittel} ist es möglich den
gemittelte Erwartungswert $\bar{\braket{I}}$ direkt zu berechnet.??
Zur Überprüfung von Gleichung \eqref{eqn:mittel} wird
der gemittelte Stromerwartungswert $\bar{\braket{I}}$
für unterschiedliche Frequenzen
$\omega\in\{\num{1,0},\num{1,5},\num{2}\}\,\frac{J}{\hbar}$
in Abhängigkeit von der Amplitude
des elektrischen Feldes
$E_0\in\left[\num{0},\num{0,1}\right]\,\frac{J}{d\symup{e}}$ auf
zwei verschieden Methoden berechnet.
Zum einen wird die Gleichung \ref{eqn:mittel}, welche die Floquet Theorie
bereitstellt, verwendet $\bar{\braket{I}}_F$ und zum anderen wird der berechnete Stromerwartungswert
über eine
Zeitspanne $T$, hier die Pulslänge $\mathcal{T}$ des Lichtfeldes,
gemittelt $\bar{\braket{I}}_M$ .
Die Abbildung \ref{fig:E_Strom} enthält jeweils die aus
den zwei verschiedenen Methoden für $a=1\,J$
berechneten gemittelten Ströme $\bar{\braket{I}}$.

\begin{figure}
  \centering
  \includegraphics[width=0.7\textwidth]{Programme/Strom_mittelwerte/Plots_mittelwerte/Potential=1.0Stromerwartungswert(t)_N=3.pdf}
  \caption{Zeitlicher Mittelwert des Stromerwartungswert $\bar{\braket{I}}$  welcher auf zweite Methoden berechnet wurde,
  für $a=1\,J$ und
  unterschiedliche Frequenzen
  $\omega\in\{\num{1,0},\num{1,5},\num{2}\}\,\frac{J}{\hbar}$
  in Abhängikeit von der Amplitude elektrischen Feldes $E_0$. }
  \label{fig:E_Strom}
\end{figure}

Aus Abbildung \ref{fig:E_Strom} tritt hervor,
dass die Gleichung \eqref{eqn:mittel} identische
Ergebnisse wie die Mittelung über eine Zeitspanne liefert.
Geringen Abweichungen sind dadurch erklärbar, dass
die Methode der Floquet Theorie den zeitlicher gemittelen Stromerwartungswert
des eingeschwungen System beschreibt, wohingegen die
Mittelung über die Pulslänge $\mathcal{T}$ Einschwingungsvorgänge des Systems
mit berücksichigt.
Somit ist bestätigt, dass der zeitlichgemittelte
Stromerwartungswert $\bar{\braket{I}}$
über die Gleichung \eqref{eqn:mittel}
berechnet werden kann.
Im Folgenden wird daher die Methode,welche die Floquet Theorie bereitstellt,
um zeitlich gemittelte Stromerwartungswerte zu berechnen,
verwendet.

\section{Überprüfung der Strom Abhängigkeiten in einem Bandisolators}
Wie in dem Kapitel \ref{sec:inverfaraday} beschrieben,
soll der Strom bei dem IFE
in einem Isolator quadratisch
mit der Amplitude des elektrischen Feldes $E_0$
sowie linear mit der Frequenz $\omega$ des elektrischen
Feldes steigen. Diese Abhängikeiten wird in diesem Abschnitt für den
Ein-Elektronen-Fall überprüft.
Für die quadratische Abhängigkeit wird in der Abbildung
\ref{fig:E_abb} der zeitliche Strommittelwert $\bar{\braket{I}}$
 gegen $E_0$ aufgetragen.
Wieder werden die Frequenzen
$\omega\in\{\num{1},\num{1,5},\num{2}\}\,\frac{J}{\hbar}$
 bei eine lokale Energie von $a=1\,J$ verwendet.
An die berechneten Werte wird versucht eine quadratische Funktion $f(x)=ax^2$
zu fitten\footnote{Mit Hilfe von python \textit{curvefit} }, ebenfalls
in Abbildung \ref{fig:E_abb} zu sehen.

\begin{figure}
  \centering
  \includegraphics[width=0.7\textwidth]{Programme/Strom_fit_e/Plots_mittelwerte/Potential=1.0Stromerwartungswert(t)_N=3.pdf}
  \caption{Zeitliche Mittelwerte des Stromerwartungswertes $\bar{\braket{I}}$  für $a=1\,J$ und
  unterschiedliche Frequenzen $\omega\in\{\num{1,0},\num{1,5},\num{2}\}\,\frac{J}{\hbar}$
  und entsprechende Fit-Funktion in Abhängikeit von der Amplitude elektrischen Feldes $E_0$. }
  \label{fig:E_abb}
\end{figure}

Die Abbildung \ref{fig:E_abb} bestätigt, dass es
möglich ist, eine quadratische Funktion an die berechneten Werte zu fitten.
Folglich ist die quadratische Abhängigkeit zu der Amplitude des elektrischen Feldes $E_0$
bestätigt.

Abschließend gilt es, die Linearität des Strommittelwertes
zu der Frequenz zu untersuchen.
Hierfür wird der Strommittelwert $\bar{\braket{I}}$
in Abhängigkeit von der Frequenz $\omega$ in einem
Bereich $\omega\in \left[0,8\right]\,\frac{J}{\hbar}$ für
zwei verschiedene elektrische Amplituden
$E_0\in\{\num{0,03},\num{0,01}\}\,\frac{J}{d\symup{e}}$
untersucht, siehe Abbildung \ref{fig:w_abb}.
Es wird eine lokale Energie von $a=1\,J$ verwendet.

\begin{figure}
   \centering
   \includegraphics[width=0.7\textwidth]{Programme/Strom_frequenzabb/Plots_mittelwerte/Potential=1.0Stromerwartungswert(t)_N=3.pdf}
   \caption{Zeitliche Mittelwerte des Stromerwartungswertes $\bar{\braket{I}}$  für $a=1J$ und
   unterschiedlichen Amplituden  $E_0\in\{\num{0,01},\num{0,01}\}\frac{J}{d\symup{e}}$
   in Abhängikeit von der Amplitude elektrischen Feldes $E_0$.}
   \label{fig:w_abb}
\end{figure}


In Abbildung \ref{fig:w_abb} werden Resonanzeffekte
bei den zuvor berechneten Resonanzfrequenzen
beobachtet. Auf Grund der
dominierenden Resonanzeffekte in Abbildung \ref{fig:w_abb}
lässt sich die Linearität nicht eindeutig bestätigen.
Um die Resonanzeffekte nicht zu betrachten, soll ein kleinerer Frequenzbereich, hier
$\omega\in\left[\num{0},\num{0,8}\right]\frac{J}{\hbar}$, untersucht werden.
Jedoch fällt der Strom in diesem Bereich unnormal
ab. Eine mögliche Ursache dafür ist eine in diesem
Frequenzbereich nicht mehr gewährleistete Orthogonalität der
QEZ, da bei einem Trunkierungsparamter von $N=3$  die Orthogonalität der QEZ nur für Frequenzen
von $\omega\geq1$ gewährleitet ist.
Um dies zu untersuchen, wird das Skalarprodukt der Quasizustände gegen
die Frequenz \omega in der Abbildung \ref{fig:N_3} für $N=3$ aufgetragen.
%Dies geschieht ebenfalls für  $N=10$, $N=20$ und $N=50$

\begin{figure}
   \centering
   \begin{subfigure}{0.48\textwidth}
       \includegraphics[width=1\textwidth]{Programme/Orthogonalitat_der_quasizustande_frequenz/Plots/Potential=1.0/Energie=0.028/Anzahl=3/Plot_fur_phi1_phi_i.pdf}
       \caption{N=3}
       \label{fig:N_3}
     \end{subfigure}
     \begin{subfigure}{0.48\textwidth}
       \includegraphics[width=1\textwidth]{Programme/Orthogonalitat_der_quasizustande_frequenz/Plots/Potential=1.0/Energie=0.028/Anzahl=10/Plot_fur_phi1_phi_i.pdf}
       \caption{N=50}
       \label{fig:N_10}
     \end{subfigure}
     \begin{subfigure}{0.48\textwidth}
       \includegraphics[width=1\textwidth]{Programme/Orthogonalitat_der_quasizustande_frequenz/Plots/Potential=1.0/Energie=0.028/Anzahl=20/Plot_fur_phi1_phi_i.pdf}
       \caption{N=50}
       \label{fig:N_20}
     \end{subfigure}
     \begin{subfigure}{0.48\textwidth}
       \includegraphics[width=1\textwidth]{Programme/Orthogonalitat_der_quasizustande_frequenz/Plots/Potential=1.0/Energie=0.028/Anzahl=50/Plot_fur_phi1_phi_i.pdf}
       \caption{N=50}
       \label{fig:N_50}
     \end{subfigure}
     \caption{Das Skalarprodukt $\braket{\Phi_1|\Phi_\alpha}$
      für unterschiedliche Trunkierparameter $N\in\{3,10,20,50\}$
      der Matrix $\mathcal{H}_F$
      in Abhängigkeit von der Frequenz $omega$
      für $a=1\, J$ und $E_0=0,03\,\frac{J}{d\symup{e}}$}
    \label{fig:N_gross}
\end{figure}


Die Abbildung \ref{fig:N_3} bestätigt
die Vermutung der fehlenden Orthogonalität.
Folglich muss der Trunkierparameter $N$ der Matrix
$\mathcal{H}_F$ erhöht werden.
Die Abbildung \ref{fig:N_gross} enthält ebenfalls
den gleichen Zusammenhang
aus \ref{fig:N_5} jedoch für
größere Parameter $N\in\{10,20,50\}$.
Es zeigt sich, dass für höhere Paramter $N$ geringere Frequenz
erreicht werden, welche die orthogonalität Bedingung
erfüllen. Durch den Antiproportional Zusammenhang.
folgt, dass geringere
Frequenzen einen hohen Trunkierungsparamter $N$
benötigen. Als folge dessen
wird die Matrix $\mathcal{H}_F$ ebenfalls
größ was sich negativ auf die Berechnungszeit
auswirkt und somit ist die Floquet Matrix Methode
für geringe Frequenzen nicht günstig.
Demzufolge ist es notwendig, bei Untersuchung des linearen Bereiches eine
größere Matrix $\mathcal{H}_F$ aufzustellen. Hier soll ein Paramter von $N=50$ genügen,
folglich dürfen Frequenzen kleiner als $0,1??$ bei der Überprüfung der Linearität nicht
berücksichtigt werden.
Der Strommittelwert $\bar{\braket{I}}$
 wird im zuvor geforderten Frequenzbereich $0-0,8$
gegen $\omega$ aufgetragen, siehe Abbildung \ref{fig:geraden_fit}.
Bei Frequenzen um $0,7??$
machen sich bereits Resonanzeffekte bemerkbar,
deshalb wird versucht eine Gerade im Frequenzbereich $0,1??$-$0,5??$
an die berechneten Werte anzupassen. ??Fitten??.

\begin{figure}
    \centering
    \includegraphics[width=0.7\textwidth]{Programme/Strom_geraden_fit/Plots_mittelwerte/Potential=1.0Stromerwartungswert(t)_N=50.pdf}
    \caption{Geraden fit}
    \label{fig:geraden_fit}
\end{figure}

Die berechneten Werte und die Gerade liegen übereinander.
Aufgrund dessen kann der lineare Zusammenhang
Zwischen der Frequenz und dem zeitlich gemittelten Stromerwartungswert
in diesem Frequenzbereich bestätigt werden.

% -berechung des Stromes
% -überprüfung der quadratischen Ahängigkeit der Stromes
% -frequenz abhängigkeit überprüfen (gerade)
% -für kleine frequenz problem mit orthogonalität

\section{Zwei-Elektronen-System}
Um einen Strommittelwert $\bar{\braket{I}}$
in dem Zwei-Elektronen-System,
welches sich im Grunzustand befindet, zu berechnen,
wird der Strommittelwert $\bar{\braket{I}}$
für ein Ein-Elektronen-System
sowohl im Grundzustand als
auch im ersten Angeregeten Zustand
berechnet. Aus Addition der beiden
Ein-Elektron-Ströme ergibt sich der Strom
des Zwei-Elektronen-System.
Die Abbildung \ref{fig:2e} enthält
den Strommittelwert des Zwei-Elektronen-Systems
in Abhänigkeit der Amplitude
des Elektrischen Feldes $E_0$.
Wieder wird das System
unterschiedliche Frequenz $\omega=??$
beobachtet.

% \begin{figure}
%     \centering
%     \includegraphics[width=0.7\textwidth]{Programme/Strom_geraden_fit/Plots_mittelwerte/Potential=0.5Stromerwartungswert(t)_N=50.pdf}
%     \caption{Zwei-Elektronen-System}
%     \label{fig:2e}
% \end{figure}
