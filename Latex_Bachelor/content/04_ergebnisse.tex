\chapter{Ergebnisse}
\label{sec:ergebnisse}
In diesem Kapitel werden zu Beginn
die im Abschnitt  \ref{sec:floquetheo} vorgestellten
Eigenschaften der Floquet Theorie am Modell überprüft.
Des Weiteren wird die zeitliche Entwicklung
eines Zustandes, welcher durch den
Floquet Formalismus berechnet wird, mit der einer
numerischen Lösung der Schrödingergleichung verglichen.
Es wird eine Methode vorgestellt, mit der es in der
Floquet Theorie möglich ist,
einen zeitlich gemittelten Strom zu berechnen.
Diese wird durch den Vergleich mit
einem über eine bestimmte Zeitspanne
gemittelten Stroms überprüft.
Anschließend wird der berechnete Strom auf die
im Abschnitt
\ref{sec:inverfaraday} prognostizierten Abhängigkeiten in einem
Isolator untersucht.
Abschließend wird das Zwei-Elektronen-System betrachtet und
ebenfalls ein gemittelter Strom berechnet.

In den folgenden Rechnungen wird der Koeffizient des Tight-Binding
Hamiltonian $J$ konventionell auf $1\si{\electronvolt}$ gesetzt \cite{czycholl}  und
für den Gitterabstand ein typischer Wert von $d=\SI{4}{\angstrom}$ gewählt.
Alle anderen Größen werden im Folgenden in Einheiten von $J$ und $d$ angegeben.
Die Bandlücke $2a$ liegt in der
Größenordnung des Tight-Binding-Koeffizienten $J$ und wird
daher auf $2\,J$ gesetzt.
Die Parameter des elektrischen Feldes werden
im Bezug auf das in \cite{jackl} beschriebene Experiment gewählt.
Die Amplitude des elektrischen Feldes liegt somit um den Wert \cite{phillip}
\begin{align}
  E_0\approx\SI{69,2}{\mega\volt\per\meter}\approx\num{0,03}\, \frac{J}{d\symup{e}}.
\end{align}
Die Frequenz des Lichtfeldes mit
einer typischen Wellenlänge $\lambda\approx\SI{800}{\nano\meter}$ ist
\begin{align}
  \omega\approx\SI{2,35}{\peta\hertz}\approx\num{1,5}\,\frac{J}{\hbar}.
\end{align}
Des Weiteren wird eine Pulsdauer
\begin{align}
 \Gamma=\SI{50}{\femto\second}\approx76\,\frac{\hbar}{J}
\end{align}
für das Lichtfeld eingeführt, in der das System beobachtet wird.
Eigenwerte und Eigenvektoren einer Matrix werden im Folgenden mit der Funktion
\textit{eig} aus dem Programm \textit{GNU Octave}
berechnet. Numerische Lösungen von der Schrödingergleichung werden
durch den adaptiven Algorithmus \textit{lsode} bestimmt, welcher ebenfalls durch das Programm
\textit{GNU Octave} bereitgestellt wird.
Zur Überprüfung der Abhängigkeiten werden verschiedene Funktionen an
die berechneten Werte gefittet.
Dies geschieht durch die Funktion \textit{curve-fit} von dem Pythonpaket \textit{scipy}.
Wichtige Naturkonstanten werden \cite{schwabl} entnommen.



\section{Frequenzabhängigkeit der Quasieigenenergien}
\label{sec:w_abb}
Zunächst werden die Eigenschaften der Floquet Theorie,
teils unabhängig von typischen Größen des Systems, untersucht. Dafür
werden die Eigenwerte
 $\epsilon_{\alpha n}$ für eine
Matrix $\mathcal{H}_\mathrm{F}$ mit Trunkierparameter $N=1$,  konstanter lokaler Energie $a=1\,J$
und elektrischer Feldamplitude $E_0=1\,\frac{J}{d\symup{e}}$
für Frequenzen in einem Bereich von $\omega\in\left[0,6\right]\,\frac{J}{\hbar}$ berechnet.
Um die Effekte des periodischen Potentials
$V(t)$ zu verdeutlichen,
werden ebenfalls Eigenwerte $E_{\alpha n}$ für eine Matrix
$\mathcal{H}_F$ mit $E_0=0\,\frac{J}{d\symup{e}}$, dies beschreibt den
zeitunabhängigen Fall, berechnet, sodass in Abbildung \ref{fig:epsilon_f}
die berechneten Eigenwerte verglichen werden können.
\begin{figure}
   \centering
   \includegraphics[width=0.7\textwidth]{Programme/Freqenzen_kontinuierlich/Plots/Plot_fur_a=1.0_E=1.0.pdf}
   \caption{Eigenwerte $\epsilon_{\alpha n}$ der Matrix $\mathcal{H}_\mathrm{F}$
    für $a=1\,J$ und $E_0=1\,\frac{J}{d\symup{e}}$ und
   Eigenwerte $E_{\alpha n}$ von $\mathcal{H}_\mathrm{F}$ für das
   zeitunabhängigen Systems $E_0=0\,\frac{J}{d\symup{e}}$
  für einem Trunkierparameter $N=1$ in Abhängigkeit von der Frequenz $\omega$.}
   \label{fig:epsilon_f}
\end{figure}
In der Abbildung \ref{fig:epsilon_f} können, wie in \cite{haenggi} beschrieben,
an Stellen, wo es zur
Überschneidung der Eigenwerte  $E_{\alpha n}$
des zeitunabhängigen Systems kommt, die Phänomene des "avoided crossing" und
"exact crossing" der QEE $\epsilon_{\alpha n}$ beobachtet werden,
die durch das zeitabhängige elektrische Feld hervorgerufen werden.
%??vielleicht??


\section{Brillouin-Zone der Quasieigenenergien}
\label{sec:E_abb}
In diesem Abschnitt wird die Darstellbarkeit der QEE, wie in
Abschnitt \ref{sec:floquetheo}
beschrieben, in der 1.
Brillouin-Zone untersucht.
Dafür werden QEE bei einer konstanten Frequenz $\omega=1\,\frac{J}{\hbar}$ und
lokaler Energie $a=1\,J$ in Abhängigkeit von der
Amplitude $E_0$, die hier zur Veranschaulichung im
Bereich $E_0\in\left[0,5\right]\,\frac{J}{d\symup{e}}$ liegt, berechnet.
Diese sind in der Abbildung \ref{fig:brillouin} gegen $E_0$ aufgetragen.
Zusätzlich wird der Trunkierparameter $N\in\{1,3,5,6\}$ der Matrix $\mathcal{H}_F$ variiert.
%
% \begin{figure}
%   \centering
%   \begin{subfigure}{0.48\textwidth}
%     \includegraphics[width=1\textwidth]{Programme/Energien_kontinuierlich/Plots/Plot_fur_a=1.0_w=1.0N=1.0.pdf}
%     \caption{Trunkierparameter $N=1$.}
%     \label{fig=N_1}
%   \end{subfigure}
%   \begin{subfigure}{0.48\textwidth}
%     \includegraphics[width=1\textwidth]{Programme/Energien_kontinuierlich/Plots/Plot_fur_a=1.0_w=1.0N=3.0.pdf}
%     \caption{Trunkierparameter $N=3.$}
%     \label{fig=N_3}
%   \end{subfigure}
%   \begin{subfigure}{0.48\textwidth}
%     \includegraphics[width=1\textwidth]{Programme/Energien_kontinuierlich/Plots/Plot_fur_a=1.0_w=1.0N=5.0.pdf}
%     \caption{Trunkierparameter $N=5$.}
%     \label{fig=N_5}
%   \end{subfigure}
%   \begin{subfigure}{0.48\textwidth}
%     \includegraphics[width=1\textwidth]{Programme/Energien_kontinuierlich/Plots/Plot_fur_a=1.0_w=1.0N=6.0.pdf}
%     \caption{Trunkierparameter $N=6$.}
%     \label{fig=N_6}
%   \end{subfigure}
%   \caption{Berechnete $\epsilon_{\alpha n}$ für unterschiedliche
%   Trunkierparameter $N\in\{1,3,5,6\}$
%   der Matrix $\mathcal{H}_F$ in Abhängigkeit von der Ampiltude $E_0$
%   bei einer lokalen Energie $a=1\,J$ und einer Frequenz
%   $\omega=1\,\frac{J}{\hbar}$. Die gestrichelten Linien grenzen dabei
%   die verschiedenen Brillouin-Zonen ab.}
%   \label{fig:brillouin}
% \end{figure}
%
Es wird deutlich, dass für kleine Trunkierparameter $N$ die Bedingung
für eine Darstellung in der 1. Brillouin-Zone nicht erfüllt ist.
Zudem wird beobachtet, dass die Übereinstimmung
der 2. Brillouin-Zone mit der 1. Brillouin-Zone mit
größeren Parameter $N$ zunimmt.
Weiterhin zeigt sich eine Abhängigkeit
zu der Amplitude $E_0$, da
größere Amplituden ebenfalls einen größeren Trunkierparameter $N$
benötigen, um die korrekten QEE zu berechnen.
Diese Beobachtungen zeigen, dass es möglich ist, die
QEE $\epsilon_\alpha$  für ein ausreichend
großen Trunkierparameter
 $N$ in die 1. Brillouin-Zone
mit $\epsilon_\alpha \in\left[-\frac{\omega}{2},\frac{\omega}{2}\right]$
 darzustellen.
Alle anderen QEE $\epsilon_{\alpha n}$ lassen sich somit
durch die Periodizitätsbedingung \eqref{eqn:epsilon_n} berechnen.
In der 1. Brillouin-Zone, siehe Abbildung \ref{fig:brillouin},
existieren genau vier QEE, folglich besitzt das System
vier QEE und vier QEZ.
Aus der Abbildung \ref{fig:brillouin} wird entnommen, dass
für Rechnungen mit einer Amplitude $E_0<1\,\frac{J}{d\symup{e}}$ ein
Trunkierparmeter $N=3$ genügt, um die korrekten QEE zu berechnen.
%Aus der  Abbildung \ref{fig:brillouin} kann entnommen werden, System vier Quasienergien besitzt

\section{Orthogonalität der Quasieigenzustände}
\label{sec:ortho}
Im Folgenden wird die Orthogonalität zwischen den
QEZ $\ket{\Phi_\alpha}$ aus der Gleichung
\eqref{eqn:ortho} für
unterschiedliche Trunkierparameter $N$ der Matrix $\mathcal{H}_F$ überprüft.
Dafür wird das Skalarprodukt
$\braket{\phi_\beta\vert\phi_\alpha}$
in Abhängigkeit von dem Parameter $N$
für die Werte $a=1\, J$, $E_0=\num{0,03}\,\frac{J}{d\symup{e}}$  und $\omega=1\,\frac{J}{\hbar}$ untersucht.
In der Abbildung \ref{fig:ortho} wird nur $\beta=1$ dargestellt, da $\beta\in\{2,3,4\}$ identische Ergebnisse liefert.
\begin{figure}
    \centering     % vielleicht mit 2x2 subfigure
    \includegraphics[width=0.5\textwidth]{Programme/Orthogonalitat_der_quasizustande/Plots/Potential=1.0/Energie=0.03/Frequenz=1.0/Plot_fur_phi1_phi_i}
    \caption{Das Skalarprodukt $\braket{\Phi_1|\Phi_\alpha}$ in Abhängigkeit von dem Trunkierparameter $N$ der Matrix $\mathcal{H}_F$ für
    $a=1\, J$ , $E_0=0,03\,\frac{J}{d\symup{e}}$  und $\omega=1\,\frac{J}{\hbar}$. }
     \label{fig:ortho}
\end{figure}
Aus Abbildung \ref{fig:ortho} wird entnommen, dass bei geringem Parameter $N$ die Orthogonalität der QEZ
nicht gegeben ist.
Für die in Abbildung \ref{fig:ortho} verwendeten Größen
erweist sich der Wert $N=3$ als ausreichend zur Gewährleistung der Orthogonalität.
Es stellt sich weiterhin ein antiproportionaler Zusammenhang von lokaler Energie
$a$ und Frequenz $\omega$ zu dem
benötigten Parameter $N$ heraus.
Wegen der Antiproportionalität ist es notwendig,
bei den folgenden Berechnungen für
Werte von $\omega<1\,\frac{J}{\hbar}$ und $a<1\,J$ die
Orthogonalität der QEZ zu überprüfen.
Andernfalls reicht eine Matrixgröße
von $N=3$ zum Erfüllen der Orthogonalitätsbedingung aus.



% -über formel .. kann quasizustand berechnet werden
% -die Orthogonalitäts bedingung formel .. sollte erfüllt sein

\section{Zeitentwicklung eines Zustandes durch den Floquet Formalismus}
\label{sec:zeit}
In diesem Abschnitt wird die Zeitentwicklung des zeitabhängigen
Ein-Elektron-Systems
durch den Floquet Formalismus bestimmt.
Als Startzustand $\ket{\Psi_0}$ wird der Grundzustand $\ket{\phi_1}$ des
zeitunabhängigen Systems verwendet, da
das System zu Beginn im Zustand niedrigster Energie verweilt.
Es wird die Zeitentwicklung $\ket{\Psi(t)}$ für eine Frequenz $\omega$ des elektrischen Feldes für ein System mit $a=1\,J$
abseits von den Resonanzfrequenzen, die sich nach Gleichung \eqref{eqn:Resonanz} berechnen,
\begin{align}
 \omega_{\text{res}_1}=\num{1,24}\,\tfrac{J}{\hbar}&  &\omega_{\text{res}_2}=\num{3,24}\,\tfrac{J}{\hbar}&  &\omega_{\text{res}_3}=\num{4,47}\,\tfrac{J}{\hbar}
\end{align}
betrachtet.
Unter dieser Bedingung wird die Frequenz $\omega=\num{1,5}\,\frac{J}{\hbar}$
und einer Amplitude von $E_0=\num{0,03}\,\frac{J}{d\symup{e}}$ gewählt.
Für diese Werte wurde
in den Abschnitten \ref{sec:E_abb} und \ref{sec:ortho} gezeigt, dass ein
Trunkierparameter von $N=3$ ausreicht, um die QEE und QEZ
korrekt zu berechnen.
%!!!!!!!!!!!!
Zur Bestimmung von $\ket{\Psi(t)}$ durch die Floquet Theorie
wird der Grundzustand durch die Gleichung \eqref{eqn:super} als als Superposition der QEZ ausgedrückt
und der Zeitpropagators \eqref{eqn:Propagator} auf diese Superposition angewendet.
% Um die Zeitentwicklung nach der Floquet Theorie
% durchzuführen, wird
% zunächst durch die Gleichung \eqref{eqn:super} der Grundzustand
% als Superposition der QEZ ausgedrückt.
% Durch Anwenden des Zeitpropagators \eqref{eqn:Propagator} auf den Grundzustand
% wird der zeitentwickelte Grundzustand
% $\ket{\Psi(t)}$ berechnet.
In Abbildung \ref{fig:zeitentwicklung} ist
die Aufenthaltswahrscheinlichkeit
$P_i(t)=\lvert\braket{i|\Psi(t)}\rvert^2$ für
die vier unterschiedlichen Gitterplätze in Abhängigkeit von der Zeit dargestellt.
Dabei wird ebenfalls die Aufenthaltswahrscheinlichkeit $P_{i_\text{num}}(t)$
einer numerischen Lösung
der Schrödingergleichung für den Grundzustand aufgetragen.
% durch den
% adaptiven Algorithmus $\textit{lsode}$,
% ??der durch das Programm Octave \cite{octave} bereitgestellt wird,?? aufgetragen.
\begin{figure}
  \centering
  \includegraphics[width=0.6\textwidth]{Programme/Eigenzustande/Plots/Potential=1.0/Energie=0.03/Besetzungen(t)_mit_Floquet_N=3w=1.5.pdf}
  \caption{Aufenthaltswahrscheinlichkeit $P$ des zeitentwickelten Grundzustandes für den $i$-ten Gitterplatz
  sowohl von dem Floquet Formalismus $P_i$ als auch von
  einer numerische Lösung $P_{i_\text{num}}$
  in Abhängigkeit von der Zeit $t$ für
  $a=1\,J$, $\omega=\num{1,5}\,\frac{J}{\hbar}$ und  $E_0=\num{0,03}\,\frac{J}{d\symup{e}}$.
  Die gestrichelten Linien markieren dabei die Periodendauer $T$ des elektrischen Feldes und
  der schwarze Balken die zu Beginn erwähnte Pulsdauer $\Gamma$ des Lichtfeldes.}
  \label{fig:zeitentwicklung}
\end{figure}
Es zeigt sich, dass die verschiedenen Lösungswege dieselben Ergebnisse liefern.
Somit ist die Richtigkeit der Zeitentwicklung in der Floquet Theorie bestätigt.
Außerdem ist erkennbar, dass die Gitterplätze $1$ und $3$
bevorzugt sind. Dies ist durch die
 Struktur des Bandisolators begründet.
 Die beiden Positionen besitzen
eine geringere lokale Energie $a$ als die anderen beiden Plätze.
%
% -wenn Zustände orthogonal zeitentwicklung möglich
% -vergleich mit lsode
% -stimmt überein eignet sich folglich für zeitentwicklung

\section{Untersuchung des Stromes in einem Bandisolator}
Die in Abschnitt \ref{sec:zeit} bestätigte Zeitentwicklung eines Zustandes durch den Floquet Formalismus
wird in diesem Abschnitt zur Berechnung des Stromflusses im System, welches sich zu Beginn im Grundzustand befindet,  genutzt.
Der Erwartungswert des Stromflusses im System ergibt sich aus dem Erwartungswert
\begin{align}
\braket{I}(t)=\braket{\psi(t)\lvert I \rvert\psi(t)}
\intertext{des Stromdichteoperators}
I= \symup{i}\frac{J}{4}\sum_i^4 \left(c_{i+1}^\dag c_{i}^{\phantom{\dag}}  -c_{i}^\dag c_{i+1}^{\phantom{\dag}}\right),
\end{align}
welcher aus der Kontinuitätsgleichung und der Heisenbergschen Bewegungsgleichung des
Teilchendichte Operators \cite{czycholl} folgt.
Werden wieder die Einheitsvektoren $\vec{e}_i$ als Basis gewählt, ergibt
sich die Matrixdarstellung des Stromdichteoperators zu
\begin{align}
I=\symup{i}\frac{J}{4}\begin{pmatrix}
  \phantom{-}0&           -1 &\phantom{-}0 & \phantom{-}1 \\
  \phantom{-}1& \phantom{-}0 &          -1 & \phantom{-}0\\
  \phantom{-}0& \phantom{-}1 &\phantom{-}0 &           -1 \\
            -1& \phantom{-}0 &\phantom{-}1 & \phantom{-}0
\end{pmatrix}.
\end{align}
Die Abbildung \ref{fig:strom_t} enthält den Erwartungswert des Stromflusses $\braket{I}(t)$ in Abhängigkeit von der Zeit.
\begin{figure}
  \centering
  \includegraphics[width=0.6\textwidth]{Programme/Strom/Plots/Potential=1.0/Energie=0.03/Stromerwartungswert(t)_N=3w=1.5.pdf}
  \caption{Erwartungswert des Stromflusses $\braket{I}(t)$ im System in Abhängigkeit der Zeit $t$ für
  $a=1\,J$, $\omega=\num{1,5}\,\frac{J}{\hbar}$ und  $E_0=\num{0,03}\,\frac{J}{d\symup{e}}$.
  Die gestrichelten Linien markieren dabei die Periodendauer $T$ des elektrischen Feldes und
  der schwarze Balken markiert die zu Beginn erwähnte Pulsdauer $\Gamma$ des Lichtfeldes.}
 \label{fig:strom_t}
\end{figure}

Da für den IFE , wie in dem Abschnitt
\ref{sec:inverfaraday} beschrieben, nur ein in der Zeit gemittelter
Stromfluss zur Magnetisierung $M$ beiträgt, wird hier der
Erwartungswert $\braket{I}$ des Stromflusses über eine Zeitspanne $T$
gemittelt. Der zeitlich
gemittelte Erwartungswert des Stromflusses $\bar{\braket{I}}$ ergibt sich aus
\begin{align}
  \bar{\braket{I}}= \frac{1}{T}\int_0^{T} \braket{\Psi(t)|I|\Psi(t)}. \label{eqn:mittelwert}
\end{align}
%Mit Hilfe des Zeitpropagators \eqref{eqn:Propagator}
In dem Floquet Formalismus ist es möglich, zeitlich
gemittelte Erwartungswerte von
Operatoren wie dem Stromdichteoperator zu berechnen.\cite{haenggi}
Durch Einsetzen von \eqref{eqn:psi_t} und \eqref{eqn:fourier}
in die Gleichung \eqref{eqn:mittelwert} ergibt sich nach Ausführung der Integration
 \begin{align}
 \bar{\braket{I}}&= \lim_{N\to\infty}\sum_\alpha \lvert c_\alpha \rvert^2  \sum_{-N}^{N} \braket{c_\alpha^n\lvert I \rvert c_\alpha^n}  \label{eqn:mittel}
 \intertext{mit}
  c_\alpha&=\braket{\Psi_0\vert\Phi_\alpha}.
\end{align}
In dem Floquet Formalismus ist es somit nicht
notwendig, erst den Erwartungswert $\braket{I}(t)$
zu bestimmen und den Mittelwert über die Gleichung \eqref{eqn:mittelwert}
zu berechnen.
%\newpage
Die Formel \eqref{eqn:mittel} ermöglicht es, den
zeitlich gemittelte Erwartungswert $\bar{\braket{I}}$ direkt zu berechnen.
Zur Überprüfung von Gleichung \eqref{eqn:mittel} wird
der zeitlich gemittelte Stromerwartungswert $\bar{\braket{I}}$
für unterschiedliche Frequenzen
$\omega\in\{\num{1,0},\num{1,5},\num{2}\}\,\frac{J}{\hbar}$
in Abhängigkeit von der Amplitude
des elektrischen Feldes
$E_0\in\left[\num{0},\num{0,1}\right]\,\frac{J}{d\symup{e}}$ auf
zwei verschiedene Methoden berechnet.
Zum einen wird die Gleichung \ref{eqn:mittel}, welche die Floquet Theorie
bereitstellt, verwendet $\bar{\braket{I}}_F$ und zum anderen wird der berechnete Stromerwartungswert
über eine
Zeitspanne $T$, hier die Pulslänge $\Gamma$ des Lichtfeldes,
gemittelt $\bar{\braket{I}}_M$ .
Die Abbildung \ref{fig:E_Strom} enthält jeweils die aus
den zwei verschiedenen Methoden für $a=1\,J$
berechneten zeitlich gemittelten Ströme $\bar{\braket{I}}$.

\begin{figure}
  \centering
  \includegraphics[width=0.6\textwidth]{Programme/Strom_mittelwerte/Plots_mittelwerte/Potential=1.0Stromerwartungswert(t)_N=3.pdf}
  \caption{Zeitlicher Mittelwert des
   Stromerwartungswertes $\bar{\braket{I}}$,
    welcher durch zwei Methoden
  für $a=1\,J$ und
  unterschiedliche Frequenzen
  $\omega\in\{\num{1,0},\num{1,5},\num{2}\}\,\frac{J}{\hbar}$
  in Abhängigkeit von der Amplitude
  des elektrischen Feldes $E_0$ berechnet wird. }
  \label{fig:E_Strom}
\end{figure}

Aus Abbildung \ref{fig:E_Strom} tritt hervor,
dass die Gleichung \eqref{eqn:mittel} identische
Ergebnisse wie die Mittelung über eine Zeitspanne liefert.
Geringe Abweichungen sind dadurch erklärbar, dass
die Methode der Floquet Theorie den im zeitlichen Limes gemittelten Stromerwartungswert
des Systems beschreibt, wohingegen eine
Mittelung über die endliche Pulslänge $\Gamma$ nur je nach Pulslänge ein um diesen Wert schwankendes
Ergebniss liefert.
Auf diese Weise ist bestätigt, dass der zeitlich gemittelte
Stromerwartungswert $\bar{\braket{I}}$
über die Gleichung \eqref{eqn:mittel}
berechnet werden kann.
Im Folgenden wird daher die Methode, welche die Floquet Theorie bereitstellt,
um zeitlich gemittelte Stromerwartungswerte zu berechnen,
verwendet.
\newpage
\section{Überprüfung der Stromabhängigkeiten in einem Bandisolator}
\label{sec:abbhangig}
Wie in dem Kapitel \ref{sec:inverfaraday} beschrieben,
soll der Strom bei dem IFE
in einem Isolator quadratisch
mit der Amplitude des elektrischen Feldes $E_0$
sowie linear mit der Frequenz $\omega$ des elektrischen
Feldes steigen. Diese Abhängigkeiten werden in diesem Abschnitt für das
Ein-Elektron-System überprüft.
Für die quadratische Abhängigkeit wird in der Abbildung
\ref{fig:E_abb} der zeitliche gemittelte Stromerwartungswert $\bar{\braket{I}}$
 gegen $E_0$ aufgetragen.
Wieder werden die Frequenzen
$\omega\in\{\num{1},\num{1,5},\num{2}\}\,\frac{J}{\hbar}$
 bei einer lokalen Energie von $a=1\,J$ verwendet.
An die berechneten Werte wird versucht, eine quadratische Funktion $f(x)=ax^2$
zu fitten, ebenfalls
in Abbildung \ref{fig:E_abb} zu sehen.

\begin{figure}
  \centering
  \includegraphics[width=0.6\textwidth]{Programme/Strom_fit_e/Plots_mittelwerte/Potential=1.0Stromerwartungswert(t)_N=3.pdf}
  \caption{Der zeitlich gemittelte Stromerwartungswert $\bar{\braket{I}}$  für $a=1\,J$ und
  unterschiedliche Frequenzen $\omega\in\{\num{1,0},\num{1,5},\num{2}\}\,\frac{J}{\hbar}$
  und entsprechende Fit-Funktion in Abhängigkeit von der Amplitude des elektrischen Feldes $E_0$. }
  \label{fig:E_abb}
\end{figure}

Die Abbildung \ref{fig:E_abb} bestätigt, dass es
möglich ist, eine quadratische Funktion an die berechneten Werte zu fitten.
Folglich ist die quadratische Abhängigkeit zu der Amplitude des elektrischen Feldes $E_0$
bestätigt.
Abschließend gilt es, die Linearität des zeitlich gemittelten
Stromerwartungswertes
zur Frequenz zu untersuchen.
Hierfür wird der zeitlich gemittelte
Stromerwartungswert $\bar{\braket{I}}$
in Abhängigkeit von der Frequenz $\omega$ in einem
Bereich $\omega\in \left[0,6\right]\,\frac{J}{\hbar}$ für
zwei verschiedene elektrische Amplituden
$E_0\in\{\num{0,01},\num{0,03},\num{0,05}\}\,\frac{J}{d\symup{e}}$
untersucht, siehe Abbildung \ref{fig:w_abb}.
Es wird eine lokale Energie von $a=1\,J$ verwendet.

\begin{figure}
   \centering
   \includegraphics[width=0.6\textwidth]{Programme/Strom_frequenzabb/Plots_mittelwerte/Potential=1.0Stromerwartungswert(t)_N=3.pdf}
   \caption{Der zeitlich gemittelte Stromerwartungswert $\bar{\braket{I}}$
  für $a=1J$, unterschiedliche Amplituden
   $E_0\in\{\num{0,01},\num{0,03},\num{0,05}\}\frac{J}{d\symup{e}}$
   in Abhängigkeit von der Frequenz des Lichtfeldes $\omega$.}
   \label{fig:w_abb}
\end{figure}
\FloatBarrier
In Abbildung \ref{fig:w_abb} werden Resonanzeffekte
bei den zuvor berechneten Resonanzfrequenzen $\omega_{\text{res}_i}$
beobachtet. Auf Grund der
dominierenden Resonanzeffekte in Abbildung \ref{fig:w_abb}
lässt sich die Linearität nicht eindeutig bestätigen.
Um die Resonanzeffekte nicht zu betrachten, wird ein
kleinerer Frequenzbereich
$\omega\in\left[\num{0},\num{0,8}\right]\,\frac{J}{\hbar}$ untersucht.
Jedoch kommt es in diesem Bereich zu
 einem zunächst nicht erklärbarem
Abfall des zeitlich gemittelten Stromerwartungswertes.
Eine mögliche Ursache dafür ist eine in diesem
Frequenzbereich nicht mehr gewährleistete Orthogonalität der
QEZ, da bei einem Trunkiersparameter von $N=3$
die Orthogonalität der QEZ nur für Frequenzen
von $\omega\geq1\,\frac{J}{\hbar}$ gewährleistet ist.
Um dies zu untersuchen, wird das Skalarprodukt der QEZ gegen
die Frequenz \omega in der Abbildung \ref{fig:N_3} für $N=3$ aufgetragen.
%Dies geschieht ebenfalls für  $N=10$, $N=20$ und $N=50$
%
% \begin{figure}
%    \centering
%    \begin{subfigure}{0.48\textwidth}
%        \includegraphics[width=1\textwidth]{Programme/Orthogonalitat_der_quasizustande_frequenz/Plots/Potential=1.0/Energie=0.028/Anzahl=3/Plot_fur_phi1_phi_i.pdf}
%        \caption{Trunkierparameter $N=3$}
%        \label{fig:N_3}
%      \end{subfigure}
%      \begin{subfigure}{0.48\textwidth}
%        \includegraphics[width=1\textwidth]{Programme/Orthogonalitat_der_quasizustande_frequenz/Plots/Potential=1.0/Energie=0.028/Anzahl=10/Plot_fur_phi1_phi_i.pdf}
%        \caption{Trunkierparameter $N=10$}
%        \label{fig:N_10}
%      \end{subfigure}
%      \begin{subfigure}{0.48\textwidth}
%        \includegraphics[width=1\textwidth]{Programme/Orthogonalitat_der_quasizustande_frequenz/Plots/Potential=1.0/Energie=0.028/Anzahl=20/Plot_fur_phi1_phi_i.pdf}
%        \caption{Trunkierparameter $N=20$}
%        \label{fig:N_20}
%      \end{subfigure}
%      \begin{subfigure}{0.48\textwidth}
%        \includegraphics[width=1\textwidth]{Programme/Orthogonalitat_der_quasizustande_frequenz/Plots/Potential=1.0/Energie=0.028/Anzahl=50/Plot_fur_phi1_phi_i.pdf}
%        \caption{Trunkierparameter $N=50$}
%        \label{fig:N_50}
%      \end{subfigure}
%      \caption{Das Skalarprodukt $\braket{\Phi_1|\Phi_\alpha}$
%       für unterschiedliche Trunkierparameter $N\in\{3,10,20,50\}$
%       der Matrix $\mathcal{H}_F$
%       in Abhängigkeit von der Frequenz $\omega$
%       für $a=1\, J$ und $E_0=\num{0,03}\,\frac{J}{d\symup{e}}$}
%     \label{fig:N_gross}
% \end{figure}
Die Abbildung \ref{fig:N_3} bestätigt
die Vermutung der fehlenden Orthogonalität.
Folglich muss der Trunkierparameter $N$ der Matrix
$\mathcal{H}_F$ erhöht werden.
Die Abbildung \ref{fig:N_gross} enthält ebenfalls
den gleichen Zusammenhang
aus \ref{fig:N_3} jedoch für
größere Parameter $N\in\{10,20,50\}$.
Es zeigt sich, dass für höhere Parameter $N$ geringere Frequenzen
erreicht werden, welche die Orthogonalitätsbedingung
erfüllen. Durch den antiproportionalen Zusammenhang
folgt, dass geringere
Frequenzen einen hohen Trunkierparameter $N$
benötigen. Als Folge dessen
wird die Matrix $\mathcal{H}_F$ ebenfalls
größer, was sich negativ auf die Berechnungsdauer
der QEE auswirkt. Demzufolge ist die Floquet Matrix Methode
für geringe Frequenzen nicht günstig.
Demzufolge ist es notwendig, bei Untersuchung des linearen Bereiches eine
größere Matrix $\mathcal{H}_F$ aufzustellen. Hier soll ein Parameter von $N=50$ genügen,
folglich dürfen Frequenzen kleiner als $\num{0,1}\,\frac{J}{\hbar}$
bei der Überprüfung der Linearität nicht
berücksichtigt werden.
Der zeitlich gemittelte Stromerwartungswert $\bar{\braket{I}}$
wird im zuvor geforderten Frequenzbereich $\omega\in\left[0,\num{0,8}\right]\,\frac{J}{\hbar}$
gegen $\omega$ aufgetragen, siehe Abbildung \ref{fig:geraden_fit}.
Bei Frequenzen um $\num{0,65}\,\frac{J}{\hbar}$
machen sich bereits Resonanzeffekte bemerkbar,
deshalb wird versucht eine Gerade im Frequenzbereich $\omega\in\left[\num{0,1},\num{0,5}\right]\,\frac{J}{\hbar}$
an die berechneten Werte zu fitten.
\begin{figure}
    \centering
    \includegraphics[width=0.7\textwidth]{Programme/Strom_geraden_fit/Plots_mittelwerte/Potential=1.0Stromerwartungswert(t)_N=50.pdf}
    \caption{Der zeitlich gemittelte Stromerwartungswert $\bar{\braket{I}}$  für $a=1\,J$,
    unterschiedliche Amplituden des elektrischen Feldes $E_0\in\{\num{0,01},\num{0,03},\num{0,05}\}\,\frac{J}{d\symup{e}}$
    und entsprechende Ausgleichsgerade in Abhängigkeit von der Frequenz $\omega$. }
    \label{fig:geraden_fit}
\end{figure}
Die berechneten Werte und die Ausgleichsgeraden liegen im Bereich $\omega\in\left[\num{0,1},\num{0,5}\right]\,\frac{J}{\hbar}$ übereinander.
Aufgrund dessen kann der lineare Zusammenhang
zwischen der Frequenz und dem zeitlich gemittelten Stromerwartungswert
in diesem Frequenzbereich bestätigt werden.

% -berechung des Stromes
% -überprüfung der quadratischen Ahängigkeit der Stromes
% -frequenz abhängigkeit überprüfen (gerade)
% -für kleine frequenz problem mit orthogonalität
\newpage
\section{Untersuchung der Stromabhängigkeiten
 im Zwei-Elektronen-System}
Um einen zeitlich gemittelten Stromerwartungswert $\bar{\braket{I}}$
in dem Zwei-Elektronen-System,
welches sich im Grundzustand befindet, zu berechnen,
wird der zeitlich gemittelte Stromerwartungswert $\bar{\braket{I}}$
für das Ein-Elektron-System
sowohl im Grundzustand als
auch im ersten angeregten Zustand
berechnet. Aus Addition der beiden
Ein-Elektron-Ströme ergibt sich der zeitlich gemittelte Stromerwartungswert
des Zwei-Elektronen-Systems.\cite{phillip}
Die Abbildung \ref{fig:2e} enthält
den zeitlich gemittelten Stromerwartungswert $\bar{\braket{I}}$ des Zwei-Elektronen-Systems
in Abhängigkeit der Amplitude
des elektrischen Feldes $E_0$.
Wie zuvor wird das System
für die Frequenzen $\omega\in\{\num{1},\num{1,5},\num{2}\}\,\frac{J}{\hbar}$
und $a=1\,J$ betrachtet.
\begin{figure}
   \centering
   \includegraphics[width=0.6\textwidth]{Programme/Zwei_elektronen/Plots_mittelwerte/Potential=1.0Stromerwartungswert(t)_N=3.pdf}
   \caption{Der zeitlich gemittelte Stromerwartungswert des Zwei-Elektronen-Systems $\bar{\braket{I}}$ für $a=1\,J$,
   unterschiedliche Frequenzen $\omega\in\{\num{1,0},\num{1,5},\num{2}\}\,\frac{J}{\hbar}$
   und entsprechender Fit-Funktion in Abhängigkeit von der Amplitude des elektrischen Feldes $E_0$. }
   \label{fig:2e}
\end{figure}

Es zeigt sich,
dass in dem betrachteten Bereich
im Zwei-Elektronen-System
keine quadratische Abhängigkeit
zu der elektrischen Amplitude
vorliegt.
% Die
% stärkere Steigung von
% $\omega=\num{1}\,\frac{J}{\hbar}$
% wird im folgenden durch die fehlende Konvergenz um $\omega=\num{1}\,\frac{J}{\hbar}$
% in den QEE und QEZ bei ein Trunkierparameter von $N=3$
% begründet. Wobei nicht klar ist weshalb wo jedoch die orthogonalitätsbedingung
% Wobei
% die stärkere Steigung von
% $\omega=\num{1}\,\frac{J}{\hbar}$
% durch ein numerischer Fehler
% fehl veruscht wird, dafür höhere
% Trunkierparameter $N$
%  Des Weiteren unterscheidet
% sich der zeitlich gemittelte Stromerwartungswert in dem Bereich
% $E_0\in\left[0,\num{0,1}\right]\,\frac{J}{d\symup{e}}$ um eine Größenordnung von $10^{-5}$ von
% dem im Ein-Elektron-System.
Wie in dem Abschnitt \ref{sec:abbhangig} wird ebenfalls
die Frequenzabhängigkeit des Stromes untersucht.
In der Abbildung \ref{fig:2_w} ist der zeitlich gemittelte
Stromerwartungswert $\bar{\braket{I}}$ gegen die Frequenz $\omega$
für $E_0\in\{\num{0,01},\num{0,03},\num{0,05}\}\,\frac{J}{d\symup{e}}$
und $a=1\,J$ aufgetragen.
\begin{figure}
   \centering
   \includegraphics[width=0.6\textwidth]{Programme/Strom_zwei_El_w_abb/Plots_mittelwerte/Potential=1.0Stromerwartungswert(t)_N=3.pdf}
   \caption{Der zeitlich gemittelte
   Stromerwartungswert $\bar{\braket{I}}$
   des Zwei-Elektronen-Systems für $a=1\,J$,
   unterschiedliche Amplituden des elektrischen Feldes $E_0\in\{\num{0,01},\num{0,03},\num{0,05}\}\,\frac{J}{d\symup{e}}$
  in Abhängigkeit von der Frequenz $\omega$. }
   \label{fig:2_w}
\end{figure}
Aus der Abbildung \ref{fig:2_w} wird entnommen, dass
der zeitlich gemittelte Stromerwartungswert des Zwei-Elektronen-Systems
deutlich geringer ist als der des Ein-Elektron-Systems.
Selbst bei den möglichen Resonanzfrequenzen $\omega_{\text{res}_i}$
des Systems weist $\bar{\braket{I}}$ keine Resonanzerscheinungen auf.
%!!!!!!!!!!!!!!!!
Wie in der zuvor betrachteten Abbildung \ref{fig:w_abb} werden bei niedrigen Frequenzen
bei denen die QEZ nicht mehr orthogonal untereinander sind numerische Fehler beobachtet.
Jedoch treten diese Fehler nicht wie durch die Abbildung \ref{fig:N_3}
für einen Trunkierparameter von $N=3$ vorrausgesagt für Frequenzen $\omega<\num{0.6}\,\frac{J}{\hbar}$
sondern bereits bei Frequenzen von $\omega\leq\num{1}\,\frac{J}{\hbar}$ auf.
Diese Beobachtung erkärt auch die stärkere Steigung von
$\omega=\num{1}\,\frac{J}{\hbar}$ in der Abbildung \ref{fig:2e} im Vergleich zu den anderen
Frequenzen. Wohin gegen die Beobachtung selber nicht mit den zuvor gewonnen Erkenntnissen
zu der Konvergenz der Lösung in der Floquet Theorie vereinbar ist.
% Wobei nicht klar ist weshalb wo jedoch die orthogonalitätsbedingung
Allerdings Zeigte sich bei genauer Betrachtung und höherem Parameter $N$,
dass $\bar{\braket{I}}$ für die untersuchten Werte von $\omega$ und $E_0$
in einer Größenordnung von $10^{-15}\,\frac{J\symup{e}}{\hbar}$
starke Schwankungen um Null besitzt,
was als numerisch $0\,\frac{J\symup{e}}{\hbar}$ interpretiert wird.
Eine mögliche Ursache für das
Fehlen von $\bar{\braket{I}}$ ist das im
Zwei-Elektronen-System wirkende Pauli-Prinzip, welches die Bewegungsfreiheit im System
der Elektronen so einschränkt, dass sich der Stromerwartungswert im zeitlichen Mittel
$\bar{\braket{I}}$ aufhebt.
