\chapter{Theoretische Grundlagen}
\label{sec:theo}
Dieses Kapitel beschäftigt sich mit den
theoretischen Grundlagen, die für diese Arbeit benötigt
werden. Zunächst werden
quantenmechanische Postulate genannt,
die hier Verwendung finden.
Darauffolgend wird eine Methode dargelegt,
den IFE in Metallen
zu beschreiben. Schließlich wird versucht,
die gewonnenen Erkenntnisse über den
IFE in Metallen auf Isolatoren zu übertragen.
Es wird eine Einführung in
die Floquet Theorie gegeben, mit der es möglich
ist, periodische quantenmechanische Systeme
zu beschreiben. Abschließend
wird eine numerische Methode vorgestellt,
um die in der Floquet Theorie beschriebenen
Größen zu berechnen.
In den folgenden Gleichungen werden
\begin{align}
   \symup{e}=\hbar=\text{m}_\symup{e}=\frac{1}{4\pi\epsilon_0}=1
\end{align}
gesetzt.




\section{Postulate der Quantenmechanik}
In der Quantenmechanik wird
ein Zustand eines Systems durch seine Wellenfunktion $\ket{\Psi}$
beschrieben. Die zeitliche Entwicklung solch
eines quantenmechanischen Systems ist durch die
zeitabhängige Schrödingergleichung
\begin{align}
\mathrm{i} \frac{\partial}{\partial t}\ket{{\Psi(t)}}=  H(t) \ket{\Psi(t)} \label{eqn:schrodinger}
\end{align}
mit dem Hamiltonoperator $H(t)$
gegeben.
Für ein zeitunabhängigen Hamiltonoperator wird durch
Separation der Variablen
\begin{align}
  \ket{\Psi(t)}=f(t)\ket{\Psi}
\end{align}
die stationäre Schrödingergleichung
\begin{align}
H \ket{\Psi}_n=E_n\ket{\Psi}_n \label{eqn:stationSG}
\end{align}
mit den Eigenenergien $E_n$ und dem
Eigenzuständen $\ket{\Psi}_n$
hergeleitet.
% Für den zeitabhängigen Teil
% \begin{align}
% \symup{i}\frac{\partial}{\partial t}f(t)=Ef(t)
% \end{align}
% er ... mach ich später
Die Wahrscheinlichkeit $P(t)$ für eine
Wellenfunktion $\ket{\Psi(t)}$
in einem
Zustand $\ket{\phi}$ ist
durch
\begin{align}
  P(t)=\lvert\braket{\phi\vert\Psi}\rvert^2
\end{align}
gegeben.
Observablen des Systems werden durch hermitesche
Operatoren dargestellt.
Der Erwartungswert $\braket{O}$ solch einer Observablen zu
dem Operator $O$
ist durch
\begin{align}
\braket{O}=\braket{\Psi\lvert O\rvert\Psi}
\end{align}
definiert.
\cite{schwabl}


\section{Inverser Faraday Effekt}
\label{sec:inverfaraday}
Der IFE beschreibt die Entstehung eines Kreisstromes
in Materie durch Bestrahlung mit zirkular
 polarisiertem Licht.
Die dabei entstehende Magnetisierung
in der Materie bleibt gegebenenfalls
auch nach Bestrahlung permanent erhalten.
Hier wird zunächst eine Herleitung des
IFE in Metallen
dargelegt und im Anschluss versucht,
die Ergebnisse auf einen Isolator
zu übertragen.
Es wird davon ausgegangen, dass
die Wechselwirkung primär zwischen dem
oszillierenden elektrischen Feld der Welle
\begin{align}
  E(x,t)=\hat{E}\exp{\left(\symup{i}kr-\symup{i}\omega t\right)}
\end{align}
und den Elektronen stattfindet.
Des Weiteren werden ein wechselwirkungsfreies Elektronenplasma angenommen und quantenmechanische Effekte vernachlässigt.
Für die Bestimmung der Stromdichte $j$,
die durch zirkular polarisiertes Licht mit der Frequenz $\omega$ und
der Amplitude $\hat{E}$ des elektrischen Feldes entsteht, dient die Kontinuitätsgleichung
\begin{align}
  \frac{\partial}{\partial t}n +\nabla(nv)=0 \label{eqn:konti}
\end{align}
mit der Elektronendichte $n$ und der Geschwindigkeit $v$
als Ausgangspunkt.
Wobei die Geschwindigkeit $v$ über den Zusammenhang
\begin{align}
  j= \symup{e}nv=\sigma E
\end{align}
mit der Stromdichte $j$, dem elektrischen Feld $E$ und
der Leitfähigkeit des bestrahlten Materials $\sigma$
verknüpft ist.
Eine Größe $a$ lässt sich unter Betrachtung von zwei unterschiedlichen Zeitskalen darstellen als
\begin{align}
  a=\braket{a}+\delta a
\end{align}
mit $\braket{a}$ dem zeitlichen Mittelwert der Größe und
$\delta a$ dem oszillierenden Anteil in Zeiteinheiten des elektrischen Feldes.
Durch Anwenden dieser Annahme auf die Größen der
Kontinuitätsgleichung \eqref{eqn:konti}
wird ein Ausdruck für die entstehende Magnetisierung $M$
\begin{align}
  \vec{M}=-\frac{\symup{i}}{4 \braket{n}\omega }\left(\sigma^*\hat{E}^* \times \sigma\hat{E}\right)\label{eqn:magnet}
\end{align}
hergeleitet.
Aus dem Zusammenhang
\begin{align}
\braket{j}=\nabla\times M \label{eqn:nabla}
\end{align}
ergibt sich die mittlere Stromdichte $\braket{j}$,
die für die Magnetisierung $M$ verantwortlich ist.
Für den Fall einer links bzw. rechts zirkular polarisierten Welle
mit der Ausbreitungsrichtung in $\vec{e}_z$ gilt
\begin{align}
  \hat E \times \hat{E}^*=\pm\symup{i}\lvert E\rvert^2\cdot \vec{e}_z.
\end{align}
Durch Einsetzen der Leitfähigkeit
für ein wechselwirkungsfreies Elektronenplasma
\begin{align}
\sigma=\frac{\symup{i}\braket{n}}{\omega}
\end{align}
in die Gleichung \eqref{eqn:magnet}
ergibt sich die Magnetisierung in einem Metall,
welches mit rechts bzw. links zirkular polarisiertem
Licht in $\vec{e}_z$-Richtung
bestrahlt wird, zu
\begin{align}
  M=\pm\frac{\symup{i}\braket{n}}{4\omega^3}\lvert\hat{E}\rvert^2 \vec{e}_z.
\end{align}
Es ist eine Antiproportionalität
zur dritten Potenz in $\omega$
und eine quadratische Abhängigkeit in $\hat{E}$
für die Magnetisierung und somit auch für den Strommittelwert
zu erkennen. \cite{hertel}

Um eine Voraussage zu den Abhängigkeiten, die
bei dem IFE in einem Bandisolator auftreten,
zu treffen, wird die Leitfähigkeit
eines Isolators
\begin{align}
  \sigma=\symup{i}\omega C
\end{align}
mit der materialabhängigen Konstante $C$
ebenfalls in die Gleichung \eqref{eqn:magnet}
eingesetzt.\cite{fließebach}
Für die Magnetisierung in einem Bandisolator, der mit
rechts bzw. links zirkular polarisiertem Licht bestrahlt wird,
ergibt sich
\begin{align}
  \vec{M}=\pm\frac{\omega C^2}{4\braket{n}} \lvert \hat{E}  \rvert^2 \vec{e}_z. \label{eqn:iso}
\end{align}
Die Gleichung \eqref{eqn:iso}
sagt somit für den IFE in Isolatoren
eine Magnetisierung
vorher, die
linear
in der Frequenz $\omega$ und
quadratisch in dem Betrag der Amplitude des elektrischen
Feldes $\lvert\hat{E}\vert$ ist.
% Dies gilt ebenfalls
% für den zeitlichen
% Strommittelwert, welcher sich aus
% der Gleichung \eqref{eqn:nabla} ergibt.



\newpage
\section{Floquet Theorie}
\label{sec:floquetheo}
In diesem Abschnitt werden die Grundlagen der
Floquet Theorie vorgestellt. Diese wird
dazu genutzt,
um ein in der Zeit periodisches quantenmechanisches System zu beschreiben.
Die zeitliche Entwicklung für so ein System
ist durch die zeitabhängige
Schrödingergleichung  \eqref{eqn:schrodinger}
gegeben. Wobei der Hamiltonian die Bedingungen
\begin{align}
  H(t)&=H(t+T),
\intertext{und}
  H(t)&=H_0+V(t)  &\text{mit}&   &V(t)=V(t+T)
\end{align}
erfüllen muss.
Er ist also periodisch
in der Zeit
und kann als Summe von einem
ungestörten Anteil $H_0$
und einem in der Zeit periodischen
Potential $V(t)$ geschrieben werden.
Das Floquet Theorem besagt, dass Zustände
 $\ket{\Psi_{\alpha}(t)}$ existieren,
welche Lösungen
von \eqref{eqn:schrodinger} sind, die
die Form
\begin{align}
\ket{\Psi(t)_\alpha}=\exp\left(-\mathrm{i}\epsilon_\alpha\right)\ket{\Phi_\alpha(t)}\label{eqn:psi_a}
\intertext{haben. Dabei ist $\ket{\Phi_\alpha(t)}$ eine sogenannte Floquet Mode und
unterliegt der periodischen Bedingung}
\ket{\Phi_\alpha(t)}=\ket{\Phi_\alpha(t+T)}.
\end{align}
Der Parameter $\epsilon_\alpha$ ist real und
wird auch charakteristischer Exponent oder
Floquet Exponent genannt.
Mit Hilfe von \eqref{eqn:psi_a} lässt sich
die Schrödingergleichung \eqref{eqn:schrodinger}
in die Form
\begin{align}
\mathcal{H}(t)\ket{\Phi_\alpha(t)}=\epsilon_\alpha \ket{\Phi_\alpha(t)} \label{eqn:floquetgl.}
\intertext{bringen, wobei der Operator }
  \mathcal{H}(t)=\left( H(t)-\mathrm{i}\frac{\partial}{\partial t} \right)
\end{align}
ist. Durch die Ähnlichkeit der Gleichung
\eqref{eqn:floquetgl.} zu der stationären
Schrödingergleichung
werden $\epsilon_\alpha$ und $\phi_\alpha$
auch als Quasieigenenergien und Quasieigenzustände,
im Folgendem abgekürzt mit QEE und QEZ,
bezeichnet.
Darüber hinaus ist
\begin{align}
  \ket{\Phi_{\alpha '}(t)}=\ket{\Phi_\alpha(t)}\exp(\mathrm{i}n\omega t) \equiv \ket{\Phi_{\alpha n}(t)}
\end{align}
ebenfalls eine Lösung von \eqref{eqn:floquetgl.},
wobei $n \in \mathbb{Z} / \{0 \} $
und $\omega$ die Frequenz des Potentials ist.
Dies hat eine Änderung der QEE in der Form
\begin{align}
    \epsilon_\alpha \rightarrow \epsilon_{\alpha '}=\epsilon_\alpha+n\omega\equiv\epsilon_{\alpha n} \label{eqn:epsilon_n}
\end{align}
zufolge.
Durch diese Feststellung ist es
möglich, alle QEE
für unterschiedliche $\alpha$
in der ersten Brillouin-Zone darzustellen,
da sich alle weiteren
QEE über \eqref{eqn:epsilon_n}
berechnen lassen.
Die QEZ unterliegen der Orthogonaliätsbedingung
\begin{align}
  \braket{\braket{\Phi_{\alpha' }(t)|\Phi_{\beta'}(t)}}\equiv \frac{1}{T} \int_0^T\mathrm{d}t
  \braket{\Phi_{\alpha' }(t)|\Phi_{\beta'}(t)}=\delta_{\alpha'\beta'}=\delta_{\alpha\beta}\delta_{nm}. \label{eqn:ortho}
\end{align}
Um die zeitliche Entwicklung
eines Startzustandes $\ket{\Psi(0)}$
mit der Floquet Theorie durchzuführen,
wird
zunächst der Startzustand $\ket{\Psi(0)}$ durch
die Superposition der QEZ
\begin{align}
  \ket{\Psi(0)}=\sum_\alpha c_a\ket{\Phi_\alpha(0)}& &\text{mit}&  &c_\alpha=\braket{\Phi_\alpha|\Psi(0)} \label{eqn:super}
\end{align}
ausgedrückt.
Der Zeitpropagator der Floquet Theorie
\begin{align}
  K(t;0)=\sum_\alpha \exp\left(-\mathrm{i}\epsilon_\alpha t \right)\ket{\Phi_\alpha(t)}\bra{\Phi_\alpha(0)} \label{eqn:Propagator}
\end{align}
ermöglicht,
den Startzustand \eqref{eqn:super}
in der Zeit zu propagieren.
Die Anwendung dieses Propagators auf den Startzustand
\eqref{eqn:super}
liefert
\begin{align}
  \ket{\Psi(t)}=K(t;0) \ket{\Psi(0)}=\sum_\alpha c_\alpha \exp\left(-\mathrm{i}\epsilon_\alpha t \right)\ket{\Phi_\alpha(t)}. \label{eqn:psi_t}
\end{align}
\cite{haenggi},\cite{dr}
% !!!!! in die Ergebnisse !!!!! Mit Hilfe dieses Propagators ist es unkomplizit in dem Floquet Formalismus, zeitlichgemittelte (read) Erwartungswerte von
% Operatoren $\hat O$ zu berechnen.
% Das zeitliche Mittel ist gegeben durch
% \begin{align}
%   \bar{\hat O}= \frac{1}{T}\int_0^T \braket{\Psi(t)|O|\Psi(t)}
% \intertext{Durch Einsetzen von \eqref{eqn:psi_t} und \eqref{eqn:fourier} ergibt sich nach Ausführung der Integration}
%  \bar{\hat O}= \sum_\alpha \lvert c_\alpha \rvert^2  \sum_{-\infty}^{\infty} c_\alpha^n(x)^\dag \hat{O} c_\alpha^n (x)^{\phantom{\dag}}.
% \end{align}
%
% Eine Herleitung dieser Formel ist im Anhang zu finden.??
\newpage
\section{Floquet Matrix Methode}
\label{sec:matrix}
Im Folgenden wird die numerische Methode
der Floquet Matrix,
welche die Berechnung der QEE
$\epsilon_{\alpha}$ und QEZ
$\ket{\Phi_\alpha}$  ermöglicht, vorgestellt.
Die QEZ $\Phi_\alpha$ lassen sich, da es
sich um in der Zeit periodische Zustände handelt,
als Fourierreihe
\begin{align}
  \ket{\Phi_{\alpha}(t)}&=\lim_{N\to\infty}\sum_{n=-N}^{N} \exp(\mathrm{i}n\omega t) \ket{c_{\alpha}^n} \label{eqn:fourier}
 \intertext{mit}
 \ket{c_\alpha^n}&=\sum_{k=1}^D c_{\alpha,k}^n \ket{\varphi_k} %phi beliebiges orthonomal set auf h_0
\end{align}
darstellen.
Hierbei ist $\ket{\varphi_k}$ ein beliebiges
orthonormales Set auf $H_0$ mit der Dimension $D$.
Durch Einsetzen von \eqref{eqn:fourier} in \eqref{eqn:floquetgl.}
ergibt sich die Definition der Matrixelemente
\begin{align}
  \braket{\bra{\varphi_j m}\mathcal{H}_\mathrm{F}\ket{\varphi_k n}}\equiv \braket{\varphi_j|H^{m-n}|\varphi_k} + n \omega \delta_{n,m}\delta_{j,k} \label{eqn:H_f}
\intertext{mit}
H^{m-n}=\frac{1}{T}\int_0^T \mathrm{d}t H(t) \exp\left(-\mathrm{i}(m-n)\omega t\right) \label{eqn:H_n_m}
\end{align}
mittels denen sich die Floquet Matrix $\mathcal{H}_\mathrm{F}$ aufstellen lässt.
Aus der Lösung der Eigenwertgleichung
\begin{align}
  \mathrm{det}|\mathcal{H}_\mathrm{F}-\epsilon\mathbb{1}|=0
\end{align}
folgen die QEE $\epsilon_{\alpha,n}$ und die Eigenvektoren $\ket{\epsilon_{\alpha,n}}$.
Diese QEE
genügen den periodischen Bedingungen aus der Gleichung \eqref{eqn:epsilon_n}.
Um aus den Eigenvektoren $\ket{\epsilon_{\alpha,n}}$
die QEZ  $\ket{\Phi_{\alpha}(t)}$
zu bestimmen, wird die Gleichung \eqref{eqn:fourier}
verwendet.
Die vektorartigen Entwicklungskoeffizienten $\ket{c_\alpha^n}$ können den
Eigenvektoren der QEE, die sich in der 1. Brillouin-Zone
befinden, entnommen werden.
Dabei entsprechen die Komponenten
der Eigenvektoren, welche zu der
$n$-ten Fourier Mode gehören,
den Komponeten $c_{\alpha,k}^n$.
\cite{haenggi},\cite{dr}
