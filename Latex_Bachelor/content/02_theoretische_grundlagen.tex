\chapter{Theoretische Grundlagen}
\label{theo}
Dieses Kapitel beschäftigt sich mit den
theoretischen Grundlagen die für diese Arbeit benötigt
werden. Zu Beginn soll der Inverse Faraday Effekt
kurz beschrieben  (ADD) und danach die Flouqet theorie
um zum Ende wird ein Numerische Methode vorgestellt
um die zuvor beschriebene Gleichung zu lösen. 

\section{Inverser Faraday Effekt}

\section{Floquet Theorie}
(READ)
Dieses Kapitel beschäftigt sich mit den Grundlagen der
Flouquet Theorie die für diese Arbeit genutzt werden soll,
um ein perodisches quantenmechanisches System zu beschreiben.

Die Zeitlicheentwicklung solch eines System ist durch die zeitabhängige Schrödinger-Gleichung
(REF)
\begin{align}
\mathrm{i}\hbar \ket{\dot{\Psi(x,t)}}=H(x,t) \  H(x,t) \ket{\Psi(x,t)} \label{eqn:schrodinger}
\end{align}
gegeben. Wobei der Hamiltionian die folgenden Bedingungen erfüllt.
Der Hamiltionian ist periodisch in der Zeit
\begin{align}
  H(x,t)=H(x,t+T)
\intertext{und lässt sich schreiben als}
H(x,t)&=H_0(x)+V(x,t)  &\text{mit}&   &V(x,t)=V(x,t+T)
\end{align}
Der Hamiltonian besteht also aus einem ungestörten Anteil $H_0(x)$ und einer in der Zeit periodischen
Störung $V(x,t)$.

Das Floquet Theorem besagt nun, dass es Zustände
 $\ket{\Psi_{\alpha}(x,t)}$ gibt, die Lösungen
 von \eqref{eqn:schrodinger} sind.
Diese Lösungen haben die Form
\begin{align}
\ket{\Psi(x,t)_\alpha}=\exp\left(-\mathrm{i}\epsilon_\alpha/ \hbar \right)\ket{\Phi_\alpha(x,t)}.\label{eqn:psi_a}
\intertext{Dabei ist $\ket{\Phi_\alpha(x,t)}$ eine sogenannte Flouqetmode und
unterliegt der periodisch Bedingung}
\ket{\Phi_\alpha(x,t)}=\ket{\Phi_\alpha(x,t+T)}.
\end{align}
Das $\epsilon_\alpha$ ist ein realer Parameter und wird auch charakteristischer Exponent oder Flouqet Exponent genannt.
Mit Hilfe von \eqref{eqn:psi_a} lässt sich die Schrödinger-Gleichung \eqref{eqn:schrodinger}
in die Form \eqref{eqn:flouqetgl.}
\begin{align}
\mathcal{H}(x,t)\ket{\Phi_\alpha(x,t)}=\epsilon_\alpha \ket{\Phi_\alpha(x,t)} \label{eqn:flouqetgl.}
\intertext{bringen, wobei der Operator }
  \mathcal{H}(x,t)=\left( H(x,t)-\mathrm{i}\hbar\frac{\partial}{\partial t} \right).
\end{align}
Durch die Ähnlichkeit von der Gleichung \eqref{eqn:flouqetgl.} zu der stationären Schrödinger
werden $\epsilon_\alpha$ und $\phi_\alpha$ auch Quasieeigenenergie und Quasieigenzustand gennannt.

Desweiteren ist
\begin{align}
  \ket{\Phi_{\alpha '}(x,t)}=\ket{\Phi_\alpha(x,t)}\exp(\mathrm{i}n\omega t) \equiv \ket{\Phi_{\alpha n}(x,t)}
\end{align}
ebenfalls eine Lösung von \eqref{eqn:flouqetgl.},  wobei $n$ einer ganzen Zahl $n=\pm1,\pm2,\pm3 \dots$
und $\omega$ die Frequenz der Störung entspricht. Dies hat aber eine Änderung der Quasienergie in der Form
\begin{align*}
    \epsilon_\alpha \rightarrow \epsilon_{\alpha '}=\epsilon_\alpha+n\hbar \omega\equiv\epsilon_{\alpha n} \label{eqn:epsilon_n}
\end{align*}
zu Folge.
Durch diese Feststellung ist es
möglich alle Quasieeigenenergien für unterschiedliche $\alpha$
in der ersten Brillouin darzustellen, da sich alle weiteren
Quasieigenenergien sich über \eqref{eqn:epsilon_n} berechnen lässen.


Die Quasizustände unterliegen der folgenden orthogornaliäts Bedingung
\begin{align}
  \braket{\braket{\Phi_{\alpha' }(t)|\phi_{\beta'}(t)}}\equiv \frac{1}{T} \int_0^T\mathrm{d}t \int_{\infty}^{\infty} \mathrm{d}x
   \braket{\braket{\Phi_{\alpha' }(t)|\phi_{\beta'}(t)}}=\delta_{\alpha'\beta'}=\delta_{\alpha\beta}\delta_{nm}.
\end{align}
(ADD)
Nun soll sich mit die Zeitlichenentwicklung der Flouqetmoden
beschäftigt werden.
Zunächst kann ein Startzustand $\ket{\Psi(0)}$ durch eine
Superposition \eqref{eqn:super} von den Flouqetmoden (READ) ausgedrückt werden, indem
\begin{align}
  \ket{\Psi(x_0,0)}=\sum_\alpha c_a\ket{\Phi_\alpha(x_0,0)},&   &c_\alpha=\braket{\Phi_\alpha|\Psi(0)}.  \label{eqn:super}.
\end{align}
Durch den Propagator\footnote{Eine genau Herleitung des Propagators $K$ ist in (REF) zufinden.}
\begin{align}
  K(x,t;x_0,0)=\sum_\alpha \exp\left(-\mathrm{i}\epsilon_\alpha/ \hbar \right)\ket{\Phi_\alpha(x,t)}\bra{\Phi_\alpha(x_0,0)} \label{eqn:zeitoperator}
\end{align}
ist es möglich den Startzustand \eqref{eqn:super}
in der Zeit zu propagieren.
Durch Anwendung von \eqref{eqn:zeitoperator} auf \eqref{eqn:super}
folgt
\begin{align}
  \Psi(x,t)=K(x,t;x_0,0) \ket{\Psi(x_0,0)}=\sum_\alpha \exp\left(-\mathrm{i}\epsilon_\alpha/ \hbar \right)\ket{\Phi_\alpha(x,t)}.
\end{align}

\section{Numerische Methoden}

Quasizustände $\Phi$ können als Fourierreihe  dargestellt werden.
\begin{align}
  \ket{\Phi_{\alpha}(x,t)}=\sum_{n=-\infty}^{\infty} c_{\alpha}^n (x)
  \exp(\mathrm{i}n\omega t).\\
\intertext{mit}
 c_\alpha^n(x)=\sum_{k=1}^\infty c_{\alpha,k}^n \ket{\varphi_k(x)} %phi beliebiges orthonomrl set auf h_0
\end{align}
