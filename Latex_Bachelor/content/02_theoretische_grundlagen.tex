\chapter{Theoretische Grundlagen}
\label{theo}
Dieses Kapitel beschäftigt sich mit den
theoretischen Grundlagen die für diese Arbeit benötigt
werden. Zu Beginn soll der Inverse Faraday Effekt
kurz beschrieben  (ADD) und danach die Floquet theorie
um zum Ende wird ein Numerische Methode vorgestellt
um die zuvor beschriebene Gleichung zu lösen.

\section{Schrödingergleichung}
Die  Zeeitlicheentwicklung eines Quantenmechnanischen ist durch die
zeitanbhängige Schrödinger-Gleichung gegeben.
Sie lautet wie folgt
(REF)
\begin{align}
\mathrm{i}\hbar \ket{\dot{\Psi(x,t)}}=H(x,t) \  H(x,t) \ket{\Psi(x,t)}. \label{eqn:schrodinger}
\end{align}
Durch Trennung der variable kann die Stationäre Schrödingergleichung
\begin{align}
E\ket{\Psi}= \hat H \ket{\Psi}
\end{align}
hergleitet werden
\section{Tight-Binding Modell?}

\section{Floquet Theorie}
(READ)
Dieses Kapitel beschäftigt sich mit den Grundlagen der
Flouquet Theorie die für diese Arbeit genutzt werden soll,
um ein perodisches quantenmechanisches System zu beschreiben.

Die Zeitlicheentwicklung solch eines System ist durch die zeitabhängige Schrödinger-Gleichung
\begin{align}
\mathrm{i}\hbar \ket{\dot{\Psi(x,t)}}=H(x,t) \  H(x,t) \ket{\Psi(x,t)} \label{eqn:schrodinger}
\end{align}
gegeben. Wobei der Hamiltionian die folgenden Bedingungen erfüllt.
Der Hamiltionian ist periodisch in der Zeit
\begin{align}
  H(x,t)=H(x,t+T)
\intertext{und lässt sich schreiben als}
H(x,t)&=H_0(x)+V(x,t)  &\text{mit}&   &V(x,t)=V(x,t+T)
\end{align}
Der Hamiltonian besteht also aus einem ungestörten Anteil $H_0(x)$ und einer in der Zeit periodischen
Störung $V(x,t)$.

Das Floquet Theorem besagt nun, dass es Zustände
 $\ket{\Psi_{\alpha}(x,t)}$ gibt, die Lösungen
 von \eqref{eqn:schrodinger} sind.
Diese Lösungen haben die Form
\begin{align}
\ket{\Psi(x,t)_\alpha}=\exp\left(-\mathrm{i}\epsilon_\alpha/ \hbar \right)\ket{\Phi_\alpha(x,t)}.\label{eqn:psi_a}
\intertext{Dabei ist $\ket{\Phi_\alpha(x,t)}$ eine sogenannte Floquetmode und
unterliegt der periodisch Bedingung}
\ket{\Phi_\alpha(x,t)}=\ket{\Phi_\alpha(x,t+T)}.
\end{align}
Das $\epsilon_\alpha$ ist ein realer Parameter und wird auch charakteristischer Exponent oder Floquet Exponent genannt.
Mit Hilfe von \eqref{eqn:psi_a} lässt sich die Schrödinger-Gleichung \eqref{eqn:schrodinger}
in die Form \eqref{eqn:floquetgl.}
\begin{align}
\mathcal{H}(x,t)\ket{\Phi_\alpha(x,t)}=\epsilon_\alpha \ket{\Phi_\alpha(x,t)} \label{eqn:floquetgl.}
\intertext{bringen, wobei der Operator }
  \mathcal{H}(x,t)=\left( H(x,t)-\mathrm{i}\hbar\frac{\partial}{\partial t} \right).
\end{align}
Durch die Ähnlichkeit von der Gleichung \eqref{eqn:floquetgl.} zu der stationären Schrödinger
werden $\epsilon_\alpha$ und $\phi_\alpha$ auch Quasieeigenenergie und Quasieigenzustand gennannt.

Desweiteren ist
\begin{align}
  \ket{\Phi_{\alpha '}(x,t)}=\ket{\Phi_\alpha(x,t)}\exp(\mathrm{i}n\omega t) \equiv \ket{\Phi_{\alpha n}(x,t)}
\end{align}
ebenfalls eine Lösung von \eqref{eqn:floquetgl.},  wobei $n$ einer ganzen Zahl $n=\pm1,\pm2,\pm3 \dots$
und $\omega$ die Frequenz der Störung entspricht. Dies hat aber eine Änderung der Quasienergie in der Form
\begin{align*}
    \epsilon_\alpha \rightarrow \epsilon_{\alpha '}=\epsilon_\alpha+n\hbar \omega\equiv\epsilon_{\alpha n} \label{eqn:epsilon_n}
\end{align*}
zu Folge.
Durch diese Feststellung ist es
möglich alle Quasieeigenenergien für unterschiedliche $\alpha$
in der ersten Brillouin darzustellen, da sich alle weiteren
Quasieigenenergien sich über \eqref{eqn:epsilon_n} berechnen lässen.


Die Quasizustände unterliegen der folgenden orthogornaliäts Bedingung
\begin{align}
  \braket{\braket{\Phi_{\alpha' }(t)|\Phi_{\beta'}(t)}}\equiv \frac{1}{T} \int_0^T\mathrm{d}t \int_{\infty}^{\infty} \mathrm{d}x
   \braket{\braket{\Phi_{\alpha' }(t)|\Phi_{\beta'}(t)}}=\delta_{\alpha'\beta'}=\delta_{\alpha\beta}\delta_{nm}.
\end{align}
(ADD)
Nun soll sich mit die Zeitlichenentwicklung der Floquetmoden
beschäftigt werden.
Zunächst kann ein Startzustand $\ket{\Psi(0)}$ durch eine
Superposition \eqref{eqn:super} von den Floquetmoden (READ) ausgedrückt werden, indem
\begin{align}
  \ket{\Psi(x_0,0)}=\sum_\alpha c_a\ket{\Phi_\alpha(x_0,0)},&   &c_\alpha=\braket{\Phi_\alpha|\Psi(0)}.  \label{eqn:super}.
\end{align}
Durch den Propagator\footnote{Eine genau Herleitung des Propagators $K$ ist in (REF) zufinden.}
\begin{align}
  K(x,t;x_0,0)=\sum_\alpha \exp\left(-\mathrm{i}\epsilon_\alpha/ \hbar \right)\ket{\Phi_\alpha(x,t)}\bra{\Phi_\alpha(x_0,0)} \label{eqn:zeitoperator}
\end{align}
ist es möglich den Startzustand \eqref{eqn:super}
in der Zeit zu propagieren.
Durch Anwendung von \eqref{eqn:zeitoperator} auf \eqref{eqn:super}
folgt
\begin{align}
  \Psi(x,t)=K(x,t;x_0,0) \ket{\Psi(x_0,0)}=\sum_\alpha \exp\left(-\mathrm{i}\epsilon_\alpha/ \hbar \right)\ket{\Phi_\alpha(x,t)}.
\end{align}

\section{Numerische Methoden}
Nun wird eine numerische Methode vorgestellt mit der es möglich ist
die Quasizustände $\Phi_\alpha$ zu berechnene, die Methode der Floquet-Matrix.
Die Quasizustände $\Phi_\alpha$ können, da es sich in der Zeit periodische Zustände handelt,
als Fourierreihe dargestellt werden.
\begin{align}
  \ket{\Phi_{\alpha}(x,t)}=\sum_{n=-\infty}^{\infty} c_{\alpha}^n (x) \label{eqn:fourier}
  \exp(\mathrm{i}n\omega t).\\
\intertext{mit}
 c_\alpha^n(x)=\sum_{k=1}^\infty c_{\alpha,k}^n \ket{\varphi_k(x)} %phi beliebiges orthonomrl set auf h_0
\end{align}
Dabei ist $\ket{\varphi_k(x)}$ ein beliebiges orthonormal Set auf $H_0$.
Durch einsetzen von \eqref{eqn:fourier} in \eqref{eqn:floquetgl.}
und weitern Umformungen(ADD) ergibt sich eine Definition für die Floquet-Matrix $\mathcal{H}_\mathrm{F}$.
Diese lautet
\begin{align}
  \braket{\bra{\varphi_j m}\mathcal{H}_\mathrm{F}\ket{\varphi_k n}}\equiv \braket{\varphi_j|H^{m-n}|\varphi_k} + n\hbar \omega \delta_{n,m}\delta_{j,k}
\end{align}
mit
\begin{align}
H^{m-n}=\frac{1}{T}\int_0^T \mathrm{d}t H(T) \exp\left(-\mathrm{i}(m-n)\omega t\right).
\end{align}
Durch lösen der Eigenwertgleichung
\begin{align}
  \mathrm{det}|\mathcal{H}_\mathrm{F}-\epsilon\mathcal{1}|=0
\end{align}
folgen die Quasieenergien $\epsilon_{\alpha,n}$ und die Eigenvektoren $\ket{\epsilon_{\alpha,n}}$.
Die Quasieenergien gehörchen den Periodischenbedingungungen aus \eqref{eqn:epsilon_n}.


\section{Inverser Faraday Effekt}
Der Inverse Faraday Effekt beschreibt die Enstehung einer Magnetisierung in Materie bei Bestrahung mit zirkularpolariesiertem Licht.
Der Inverse Faraday Effekt beschreibt die Enstehung einer Magnetisierung in Materie, die mit zirkularpolariesiertem Licht bestrahlt wird.
Dabei wird davon ausgegangen, dass die Wechselwirkung primär zwischen dem E-felder der Welle und den Elektronen statt findet.
Es kann durch mehrere Überlegungen gezeigt werden, dass der Strom, der durch das zirkularpolarisierte Licht erzeugt wird, eine quadratische Abhängigkeit
zu der E-feld Amplitude besitzt.
