\chapter{Theoretische Grundlagen}
\label{sec:theo}
Dieses Kapitel beschäftigt sich mit den
theoretischen Grundlagen die für diese Arbeit benötigt
werden. Zunächst werden
Quantenmechnaischen Postulate genannt,
die hier Verwendung finden.
Darauffolgend wird eine Mehtode dargellegt
um den inverse Faraday Effekt in Metallen
zu beschreiben. Schließlich wird versucht
die gewonnene Erkenntisse über den
IFE auf Isolatoren zu übertragen.
Abschießend wird eine Einführung in
die Floquet Theorie, mit der es möglich
ist periodische Quantenmechanische Systeme
zu beschreiben, gegeben
und eine numerische Methode vorgestellt,
um die, in der Floquet Theorie beschriebenen,
Probleme(/Gleichungen)  zu lösen.
In den folgenen Gleichungen werden
\begin{align}
   \symup{e}=\hbar=\text{m}_\symup{e}=\frac{1}{4\pi\epsilon_0}=1.
\end{align}
gesetzt.




\section{Postulate der Quantenmechanik}
In der Quantenmechanik wird
ein Zustand eines Systems durch seine Wellenfunktion $\ket{\Psi}$
beschrieben. Die zeitliche Enwicklung solch
eines quantenmechanischen Systems ist durch die
zeitabhängige Schrödingergleichung
\begin{align}
\mathrm{i} \ket{\dot{\Psi(\vec{x},t)}}=  H(\vec{x},t) \ket{\Psi(\vec{x},t)} \label{eqn:schrodinger}
\end{align}
mit der Hamiltonoperator $H(\vec{x},t)$
gegeben.
Für ein zeitunabhängigen Hamiltonoperator wird durch
Seperation der Variablen
\begin{align}
  \Psi(x,t)=f(t)\Phi(x)
\end{align}
die stationäre Schrödingergleichung
\begin{align}
\hat H \ket{\Phi(x)}_n=E_n\ket{\Phi(x)}_n \label{eqn:stationSG}
\end{align}
mit den Eigenenergie $E_n$ und dem
Eigenzuständen $\ket{\Phi(x)}_n$
hergleitet.
??Für den zeitabhängigen Teil
\begin{align}
\symup{i}\frac{\partial}{\partial t}f(t)=Ef(t)
\end{align}
??
Die Wahrscheinlichkeit $P$ für eine
Wellenfunktion $\ket{\Psi}$ in einem
Zustand $\ket{\phi}$ ist
durch
\begin{align}
  P=\lvert\braket{\phi\vert\Psi}\rvert^2
\end{align}
gegeben.
?Die? Observablen des Systems werden durch hermitesche
Operatoren dargestellt.
Der Erwartungswert $\braket{O}$ solch einer Observablen zu
dem Operator $O$
ist durch
\begin{align}
\braket{O}=\braket{\Psi\lvert O\rvert\Psi}
\end{align}
definiert.
\cite{schwabl}


\section{Inverser Faraday Effekt}
\label{sec:inverfaraday}
Der inverse Faraday Effekt beschreibt die Entstehung eines Kreisstromes
in Materie durch Bestrahlung mit zirkular polarisiertem Licht.
Die dabei entstehende Magnetisierung in der Materie kann gegebenenfalls
auch nach Bestrahlung permanent erhalten bleiben.

Hier wird zunächst eine Herleitung des inversen Faraday Effekt in Metallen
dargelegt und im Anschluss versucht die Ergebnisse auf einen Isolators
zu übertragen.

Es wird davon ausgegangen, dass die Wechselwirkung primär zwischen dem
oszillierenden elektrischen Feld der Welle
\begin{align}
  E(x,t)=\hat{E}\exp{\left(\symup{i}kr-\symup{i}\omega t\right)}
\end{align}
und den Elektronen stattfindet.
Des weiteren wird ein wechselwirkungsfreies Elektronenplasma angenommen und quantenmechanische Effekte vernachlässigt.
Für die Bestimmung der Stromdichte $j$,
die durch zirkualarpolarisiertes Licht mit der Frequenz $\omega$ und
der Amplitude $\hat{E}$ des elektrischen Feldes entsteht, dient die Kontinuitätsgleichung
\begin{align}
  \frac{\partial}{\partial t}n +\nabla(nv), \label{eqn:konti}
\end{align}
mit der Elektronendichte $n$ und der Geschwindigkeit $v$,
als Ausgangspunkt.
Wobei die Geschwindigkeit $v$ über den Zusammenhang
\begin{align}
  j=env=\sigma E
\end{align}
mit der Stromdichte $j$, dem elektrischen Feld $E$ und
der Leitfähigkeit des bestrahlten Materials $\sigma$.\\
verknüpft ist.
Eine Größe $a$ lässt sich unter Betrachtung von zwei unterschiedlichen Zeitskalen darstellen als
\begin{align}
  a=\braket{a}+\delta a,
\end{align}
mit $\braket{a}$ dem zeitlichen Mittelwert der Größe und
$\delta a$ dem oszillierenden Anteil in Zeitenheiten des elektrischen Feldes.

Durch Anwenden dieser Annahme auf die Größen der Kontinuitätsgleichung \eqref{eqn:konti}
wird ein Ausdruck für die entstehende Magnetisierung $M$
\begin{align}
  \vec{M}=-\frac{\symup{i}}{4 \braket{n}\omega }\left(\sigma^*\hat{E}^* \times \sigma\hat{E}\right).\label{eqn:magnet}
\end{align}
hergeleitet.
Aus dem Zusammenhang
\begin{align}
\braket{j}=\nabla\times M
\end{align}
ergibt sich der mittlere Strom $\braket{j}$
der für die Magnetisierung $M$ verantwortlich ist.
Für den Fall einer linkes bzw. recht zirkular polarisierten Welle
mit der Ausbreitungsrichtung in $e_z$, gilt
\begin{align}
  \hat E \times \hat{E}^*=\pm\symup{i}\lvert E\rvert^2\cdot e_z.
\end{align}
Durch Einsetzen der Leitfähigkeit
für ein wechselwirkungsfreies Elektronenplasma
\begin{align}
\sigma=\frac{\symup{i}\braket{n}}{\omega}
\end{align}
in die Gleichung \eqref{eqn:magnet}
ergibt sich die Magnetisierung in einem Metall,
welches mit rechts bzw. links zirkular polarisiertem
Licht in $\vec{e}_z$-Richtung
bestrahlt wird, zu
\begin{align}
  M=\pm\frac{\symup{i}\braket{n}}{4\omega^3}\lvert\hat{E}\rvert^2 \vec{e}_z
\end{align}
Es ist eine antiproportionalität
zur dritten Potenz in \omega
und eine quadratische Abhänigkeit in $\hat{E}$
für die Magnetierung und somit auch den Strommittelwert
zu erkennen.

Um eine Vorraussage zu dem Abhängigkeiten, die
bei dem IFE in einem Bandisolatoren auftreten,
zu treffen, wird die Leitfähigkeit
eines Isolators \cite{fließebach}
\begin{align}
  \sigma=\symup{i}\omega C
\end{align}
mit der materialabhängigen Konstate $C$
ebenfalls in die Gleichung \eqref{eqn:magnet}
eingesetzt.
Für die Magnetisierung in einem Bandisolator, der mit
recht bzw. links polariertem Licht bestrahlt wird,
ergibt sich
\begin{align}
  \vec{M}=\pm\frac{\omega C}{4\braket{n}} \lvert \hat{E}  \rvert^2 \vec{e}_z.
\end{align}
Es zeigt sich,
dass der Strommittelwert und die
Magnetisierung, erzeugt durch IFE in einem Isolator, linear
mit der Frequenz \omega und
quadratische mi E-Feld-Amplitude besitzt.




\section{Floquet Theorie}
\label{sec:floquetheo}
In diesem Abschnitt werden die Grundlagen der
Floquet Theorie vorgestellt. Diese kann genutzt werden,
um ein in der Zeit periodisches quantenmechanisches System zu beschreiben.

Die zeitliche Entwicklung für so ein System
ist durch die zeitabhängige
Schrödingergleichung  \eqref{eqn:schrodinger}
gegeben. Wobei der Hamiltonian die Bedingungen
\begin{align}
  &H(t)=H(t+T),
\intertext{und}
  H(t)&=H_0+V(t)  &\text{mit}&   &V(t)=V(t+T)
\end{align}
erfüllen muss.
Der Hamiltonian ist also periodisch in der Zeit und
und kann als Summe von einem
ungestörten Anteil $H_0$
 und einer in der Zeit periodischen
Störung $V(t)$ geschrieben werden.

Das Floquet-Theorem besagt, dass Zustände
 $\ket{\Psi_{\alpha}(t)}$ exsitieren,
welche Lösungen
von \eqref{eqn:schrodinger} sind, die
die Form
\begin{align}
\ket{\Psi(t)_\alpha}=\exp\left(-\mathrm{i}\epsilon_\alpha\right)\ket{\Phi_\alpha(t)}.\label{eqn:psi_a}
\intertext{bestizen(haben). Dabei ist $\ket{\Phi_\alpha(x,t)}$ eine sogenannte Floquetmode und
unterliegt der periodischen Bedingung}
\ket{\Phi_\alpha(t)}=\ket{\Phi_\alpha(t+T)}.
\end{align}
Der Parameter $\epsilon_\alpha$ ist real und
wird auch charakteristischer Exponent oder
Floquet Exponent genannt.
Mit Hilfe von \eqref{eqn:psi_a} lässt sich
die Schrödingergleichung \eqref{eqn:schrodinger}
in die Form
\begin{align}
\mathcal{H}(t)\ket{\Phi_\alpha(t)}=\epsilon_\alpha \ket{\Phi_\alpha(t)} \label{eqn:floquetgl.}
\intertext{bringen, wobei der Operator }
  \mathcal{H}(t)=\left( H(t)-\mathrm{i}\frac{\partial}{\partial t} \right).
\end{align}
ist. Durch die Ähnlichkeit der Gleichung
\eqref{eqn:floquetgl.} zu der stationären
Schrödingergleichung
werden $\epsilon_\alpha$ und $\phi_\alpha$
auch als Quasieigenenergien und Quasieigenzustände,
im Folgendem Abgekürzt mit QEE und QEZ,
bezeichnet.

Darüber hinaus ist
\begin{align}
  \ket{\Phi_{\alpha '}(t)}=\ket{\Phi_\alpha(t)}\exp(\mathrm{i}n\omega t) \equiv \ket{\Phi_{\alpha n}(t)}
\end{align}
ebenfalls eine Lösung von \eqref{eqn:floquetgl.},
wobei $n \in \mathbb{Z} / \{0 \} $
und $\omega$ die Frequenz der Störung ist.
Dies hat eine Änderung der Quasienergie in der Form
\begin{align}
    \epsilon_\alpha \rightarrow \epsilon_{\alpha '}=\epsilon_\alpha+n\omega\equiv\epsilon_{\alpha n} \label{eqn:epsilon_n}
\end{align}
zufolge.
Durch diese Feststellung ist es
möglich, alle Quasieigenenergien
für unterschiedliche $\alpha$
in der ersten Brillouin-Zone darzustellen,
da sich alle weiteren
Quasieigenenergien über \eqref{eqn:epsilon_n}
berechnen lassen.


Die Quasizustände unterliegen der Orthogonaliätsbedingung
\begin{align}
  \braket{\braket{\Phi_{\alpha' }(t)|\Phi_{\beta'}(t)}}\equiv \frac{1}{T} \int_0^T\mathrm{d}t
  \braket{\Phi_{\alpha' }(t)|\Phi_{\beta'}(t)}=\delta_{\alpha'\beta'}=\delta_{\alpha\beta}\delta_{nm}. \label{eqn:ortho}
\end{align}
\\
Um die zeitliche Entwicklung
eines Startzustandes $\ket{\Psi(0)}$
mit der Floquet Theorie durchzuführen,
wird
zunächst der Startzustand $\ket{\Psi(0)}$ durch
die Superposition der Floquetmoden
\begin{align}
  \ket{\Psi(0)}=\sum_\alpha c_a\ket{\Phi_\alpha(0)}& &\text{mit}&  &c_\alpha=\braket{\Phi_\alpha|\Psi(0)}.  \label{eqn:super}
\end{align}
ausgedrückt.
Durch den Zeitpropagator\footnote{Eine genau Herleitung des Propagators $K$ ist in (REF) zu finden.}
\begin{align}
  K(t;0)=\sum_\alpha \exp\left(-\mathrm{i}\epsilon_\alpha t \right)\ket{\Phi_\alpha(t)}\bra{\Phi_\alpha(0)} \label{eqn:Propagator}
\end{align}
ist es möglich, den Startzustand \eqref{eqn:super}eqn:Propagator
in der Zeit zu propagieren.
Anwenden dieses Propagators auf den Startzustnad
\eqref{eqn:super}
liefert
\begin{align}
  \Psi(t)=K(t;0) \ket{\Psi(0)}=\sum_\alpha c_\alpha \exp\left(-\mathrm{i}\epsilon_\alpha t \right)\ket{\Phi_\alpha(t)}. \label{eqn:psi_t}
\end{align}
% !!!!! in die Ergebnisse !!!!! Mit Hilfe dieses Propagators ist es unkomplizit in dem Floquet-Formalismus, zeitlichgemittelte (read) Erwartungswerte von
% Operatoren $\hat O$ zu berechnen.
% Das zeitliche Mittel ist gegeben durch
% \begin{align}
%   \bar{\hat O}= \frac{1}{T}\int_0^T \braket{\Psi(t)|O|\Psi(t)}
% \intertext{Durch Einsetzen von \eqref{eqn:psi_t} und \eqref{eqn:fourier} ergibt sich nach Ausführung der Integration}
%  \bar{\hat O}= \sum_\alpha \lvert c_\alpha \rvert^2  \sum_{-\infty}^{\infty} c_\alpha^n(x)^\dag \hat{O} c_\alpha^n (x)^{\phantom{\dag}}.
% \end{align}
%
% Eine Herleitung dieser Formel ist im Anhang zu finden.??

\section{Numerische Methoden/ Floquet Matrix Methode}
\label{sec:matrix}
Im Folgenden wird die numerische Methode
der Floquet Matrix,
welche die Berechnung der Quasieigenenenergien
$\epsilon_{\alpha}$ und Quasieigenzustände
$\ket{\Phi_\alpha}$  ermöglicht, vorgestellt.
Die Quasizustände $\Phi_\alpha$ können, da es
sich um in der Zeit periodische Zustände handelt,
als Fourierreihe
\begin{align}
  \ket{\Phi_{\alpha}(t)}=\lim_{N\to\infty}\sum_{n=-N}^{N} \exp(\mathrm{i}n\omega t) \ket{c_{\alpha}^n} \label{eqn:fourier}
\intertext{mit}
 \ket{c_\alpha^n}=\sum_{k=1}^D c_{\alpha,k}^n \ket{\varphi_k} %phi beliebiges orthonomal set auf h_0
\end{align}
dargestellt werden.
Hierbei ist $\ket{\varphi_k}$ ein beliebiges
orthonormales Set auf $H_0$??, welcher die Dimension $D$ besitzt??.
Durch Einsetzen von \eqref{eqn:fourier} in \eqref{eqn:floquetgl.}

ergibt sich \cite{haggi} die Definition der Matrixelemente
\begin{align}
  \braket{\bra{\varphi_j m}\mathcal{H}_\mathrm{F}\ket{\varphi_k n}}\equiv \braket{\varphi_j|H^{m-n}|\varphi_k} + n \omega \delta_{n,m}\delta_{j,k} \label{eqn:H_f}
\intertext{mit}
H^{m-n}=\frac{1}{T}\int_0^T \mathrm{d}t H(t) \exp\left(-\mathrm{i}(m-n)\omega t\right) \label{eqn:H_n_m}
\end{align}
mittels denen sich die Floquet Matrix $\mathcal{H}_\mathrm{F}$ aufstellen lässt.
Aus der Lösung der Eigenwertgleichung
\begin{align}
  \mathrm{det}|\mathcal{H}_\mathrm{F}-\epsilon\mathbb{1}|=0
\end{align}
folgen die Quasienergien $\epsilon_{\alpha,n}$ und die Eigenvektoren $\ket{\epsilon_{\alpha,n}}$.
Diese Quasienergien genügen den periodischen Bedingungen aus \eqref{eqn:epsilon_n}.
Um aus den Eigenvektoren $\ket{\epsilon_{\alpha,n}}$
die Quasizustände  $\ket{\Phi_{\alpha}(t)}$
zu bestimmen, wird die Gleichung \label{eqn:fourier}
verwendet.
Die vektorartigen Entwicklungskoeffizienten $\ket{c_\alpha^n}$ können den
Eigenvektoren der QEE, die sich in der 1. Brillouin
befinden, entnommen werden.
Dabei entsprechen die Komponenten
der Eigenvektor, welche zu der
$n$-ten Fouiermode gehören,
den Komponeten $c_{\alpha,k}^n$.
