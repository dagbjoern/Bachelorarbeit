\chapter{Einleitung}
Hier folgt eine kurze Einleitung in die Thematik der Bachelorarbeit.
Die Einleitung muss kurz sein, damit die vorgegebene Gesamtlänge der
Arbeit von 25 Seiten nicht überschritten wird.
Die Beschränkung der Seitenzahl sollte man ernst nehmen,
da Überschreitung zu Abzügen in der Note führen kann.
Um der Längenbeschränkung zu genügen, darf auch nicht an der Schriftgröße,
dem Zeilenabstand oder dem Satzspiegel (bedruckte Fläche der Seite) manipuliert werden.


% hier sollten Verweise auf die Paper stehen wieso inverser Farady effekt interessant
%und  wieso das Floquet theorem von Nutzen ist.



\section{Inverser Faraday Effekt}
Der Inverse Faraday Effekt beschreibt die Enstehung einer Magnetisierung in Materie bei Bestrahung mit zirkularpolariesiertem Licht.
Der Inverse Faraday Effekt beschreibt die Enstehung einer Magnetisierung in Materie, die mit zirkularpolariesiertem Licht bestrahlt wird.
Dabei wird davon ausgegangen, dass die Wechselwirkung primär zwischen dem E-felder der Welle und den Elektronen statt findet.
Es kann durch mehrere Überlegungen gezeigt werden, dass der Strom, der durch das zirkularpolarisierte Licht erzeugt wird, eine quadratische Abhängigkeit
zu der E-feld Amplitude besitzt.
