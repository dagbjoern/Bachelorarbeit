\chapter{Einleitung}
In dieser Arbeit wird versucht,
den inversen Faraday Effekt
in einem Bandisolator mit
Hilfe der Floquet Theorie
zu beschreiben.

Der Inverse Farady Effekt (IFE) stellt eine Möglichkeit bereit,
die Magnetisierung in einem Material
durch Bestrahlung zu beeinflussen,
somit bietet er eine interessante
Methode für die
Kontrolle von Magnetisierungsprozessen
 in Nanostrukturen.
Durch ihn werden beispielsweise, wie schon
in Experimenten gezeigt \cite{jackl},
Spin-Wellen in einem Material erzeugt.
Weiterhin bietet er interessante
Aspekte für die Informationstechnik \cite{hertel}).
Als Grundlage für die
theoretische Beschreibung des IFE dient hier
der Artikel "Theory of the inverse Faraday effect
in metals" \cite{hertel} von Riccardo
Hertel, der sich mit dem IFE in Metallen
beschäftigt.

Die Floquet Theorie liefert eine Methode
zur nährungsweisen Berschreibung
getriebener quantenmechanischer Systeme.
Ein Vorteil dieser Theorie ist
die Berücksichtigung der Periodizität
der Störung auf jedem Nährungsschritt.
Ebenfalls ist das Fehlen von
säkularen Termen, die linear oder nicht
periodisch in der Zeit sind, von Vorteil
im Floquet Formalismus.\cite{haenggi}
Sie wird zum Beispiel in dem Artikel \cite{mentink}
dazu genutzt, die Austauschwechselwirkung
in Mott Isolatoren zu untersuchen.
Das Kapitel
"Driven Quantum Systems"
von Peter Hänggi aus dem Buch \cite{haenggi}
dient als Basis für die hier verwendete Floquet Theorie.


In dem Kapitel \ref{sec:theo} werden die
hier benötigten theoretischen Grundlagen
sowohl für den IFE als auch für die Floquet Theorie aufgeführt.
Anschließend wird in Kapitel \ref{sec:model} ein
Modell eines Bandisolators aufgestellt,
in dem der IFE untersucht wird.
Die durch das Modell gewonnenen Ergebnisse
werden in dem Kapitel \ref{sec:ergebnisse} dargelegt.
Abschließend werden in Kapitel \ref{sec:zusamm}
noch einmal die Vorgehensweise
und die gewonnenen Erkenntisse der vorherigen
Kapitel \ref{sec:model} und \ref{sec:ergebnisse}
zusammmengefasst und ein Ausblick
auf noch offene Aspekte gegeben.

%Im Rahmen dieser Bachelorarbeit wird sich mit der Thematik des Inversen Faraday Effektes (IFE) beschäftigt.
%Der IFE bietet eine Möglichkeit zur Kontrolle von ulta schnellen magnetisierungs Prozessen in Nanostrukturen

%sowie die kontrolle von Spinwellen in Materialien   blablabla ist.
%Welcher mit Hilfe der Floquet-Theorie untersucht werden soll.





% hier sollten Verweise auf die Paper stehen wieso inverser Farady effekt interessant
%und  wieso das Floquet theorem von Nutzen ist.

%
%
% \section{Inverser Faraday Effekt}
% Der Inverse Faraday Effekt beschreibt die Enstehung eines Kreisstromes in Materie, welcher
% eine Magnetisierung in dem Material hervorruft, bei Bestrahung mit zirkularpolariesiertem Licht.
%
% Der Inverse Faraday Effekt beschreibt die Enstehung einer Magnetisierung in Materie, die mit zirkularpolariesiertem Licht bestrahlt wird.
% Dabei wird davon ausgegangen, dass die Wechselwirkung primär zwischen dem E-felder der Welle und den Elektronen statt findet.
% Es kann durch mehrere Überlegungen gezeigt werden, dass der Strom, der durch das zirkularpolarisierte Licht erzeugt wird, eine quadratische Abhängigkeit
% zu der E-feld Amplitude besitzt.
