\chapter{Einleitung}
Der Inverse Farady Effekt (IFE) bietet eine Möglichkeit
zur Kontrolle von Magnetisierungsprozessen in Nanostrukturen.
Durch ihn können Beispielsweise, wie schon in Experimenten gezeigt \cite{E2},
Spin-Wellen in einem Material erzeugt und kontrolliert werden.
Ebenfalls könnte er ??irgendwann?? zur Datenspeicherung genutzt werden \cite{}.
Als Grundlage für den IFE wird der Artikel
"Theory of the inverse Faraday effect
 in metals" \cite{hertel} von Riccardo
Hertel, der sich mit dem IFE im Metallen
beschäftigt, genutzt.

Die Floquet-Theorie liefert eine Methode
getriebene quantenmechanische Systeme
Nährungsweise zu beschreiben.
In blabla wurde Floquet angewendet .
!!Wobei die Periodizität
der Störung auf jedem Nährungsschritt
berücksichtigt wird.!!(vielleicht zu stark abgeschrieben)
Ein weiterer Vorteil besteht
durch das nicht Auftreten von
sekular Termen, die linear oder nicht
periodisch in der Zeit sind.
Der Artikel von Peter Hänggi
"Driven Quantum Systems" \cite{haggi}
dient als Basis für die hier verwendete Floquet-Theorie.


In dieser Arbeit wird versucht
den inversen Faraday Effekt
in einem Bandisolator mit
Hilfe der Floquet-Theorie
zu beschreiben.
Dafür werden zunächst benötigte
theoretische Grundlagen (sowohl für den IFE als auch der Floquet-Theorie) aufgeführt.
Anschließend wird ein Modell eines Bandisolator aufgestellt,
indem der inverse Faraday Effekt untersucht wird.


%Im Rahmen dieser Bachelorarbeit wird sich mit der Thematik des Inversen Faraday Effektes (IFE) beschäftigt.
%Der IFE bietet eine Möglichkeit zur Kontrolle von ulta schnellen magnetisierungs Prozessen in Nanostrukturen

%sowie die kontrolle von Spinwellen in Materialien   blablabla ist.
%Welcher mit Hilfe der Floquet-Theorie untersucht werden soll.





% hier sollten Verweise auf die Paper stehen wieso inverser Farady effekt interessant
%und  wieso das Floquet theorem von Nutzen ist.

%
%
% \section{Inverser Faraday Effekt}
% Der Inverse Faraday Effekt beschreibt die Enstehung eines Kreisstromes in Materie, welcher
% eine Magnetisierung in dem Material hervorruft, bei Bestrahung mit zirkularpolariesiertem Licht.
%
% Der Inverse Faraday Effekt beschreibt die Enstehung einer Magnetisierung in Materie, die mit zirkularpolariesiertem Licht bestrahlt wird.
% Dabei wird davon ausgegangen, dass die Wechselwirkung primär zwischen dem E-felder der Welle und den Elektronen statt findet.
% Es kann durch mehrere Überlegungen gezeigt werden, dass der Strom, der durch das zirkularpolarisierte Licht erzeugt wird, eine quadratische Abhängigkeit
% zu der E-feld Amplitude besitzt.
