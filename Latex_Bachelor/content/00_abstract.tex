\thispagestyle{plain}

\section*{Kurzfassung}
Diese Bachelorarbeit behandelt eine Möglichkeit
den inversen Faraday Effekt
in einem Bandisolator
mittels Floquet Theorie zu beschreiben.
Ein (zwei dimensionales)/(2D) Modell eines Bandisolators
wurde aufgestellt und aus diesem
eine Floquet Matix abgeleitet.
Es stellte sich heraus, dass es, zur
korrekten Berschreibung des Systems,
notwendig war
sowohl
für geringe Frequenzen der Bestrahlung
als auch für geringer lokale Energien
im Isolator
höhere Fourier Moden
in der Flouqet Matrix zu berücksichtigen.
Desweiteren wurde in dem System, welches
ein Elektron enthält, ein Strommittelwert
berechnet. Der Strommittelwert weiste
 (die vorhergesagt )/(eine)
Lineare Abhängigkeit
zu der Frequnez sowie eine quadratische Abhängigkeit
zu der elektrischen Amplitude der Lichtquelle auf.
Bei der Überprüfung für ein Zwei-Elektronen-System,
konnte der zuvor gefundene Zusammenhang
jedoch nicht bestätigt werden.
Somit ist eine andere Theorie notwendig,
um den inversen Faraday Effekt
in einem Bandisolator genau zu beschreiben.


\section*{Abstract}
\begin{english}
This bachelor thesis deals with the possibility to describe
the inverse Faraday effect a Band-insulator by Floquet theory.
Therefore a model of a Band-insulator was deduced.
It turned out that was necessary, for the
correct description of the system, to regard (the)
higher Fourier modes in the Floquet matrix  for lower frequencies
as well as lower local energies. Futhermore the
avarage current for a one electron was calculated.
This avarage current showed a linear relation to the
frequency and a quadratic relation to the electic amplitude
of the light source. These relations were not
confirmed for a two
electron system.
As a result another theory is necessery to describe the IFA
in a Band-insulator exactly
\end{english}
