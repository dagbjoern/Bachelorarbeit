\thispagestyle{plain}

\section*{Kurzfassung}
Diese Bachelorarbeit behandelt eine Möglichkeit,
den inversen Faraday Effekt
in einem Bandisolator
mittels der Floquet Theorie zu beschreiben.
Ein 2D Modell eines Bandisolators,
welcher spinlose Elektronen enthält,
wurde aufgestellt und aus diesem
eine Floquet Matrix abgeleitet.
Es stellte sich heraus, dass es, zur
korrekten Berschreibung des Systems,
notwendig war,
für geringe Frequenzen, hohe
elektrische Amplituden des Lichtfeldes
und geringe lokale Energien
im Isolator
höhere Fourier Moden
in der Flouqet Matrix zu berücksichtigen.
Des Weiteren wurde in dem System, welches
ein Elektron enthält, ein zeitlicher Strommittelwert
berechnet. Der Strommittelwert wies
die prognostizierte
lineare Abhängigkeit
zu der Frequenz sowie eine quadratische Abhängigkeit
zu der elektrischen Amplitude der Lichtquelle auf.
Bei der Überprüfung für ein Zwei-Elektronen-System
konnte der zuvor gefundene Zusammenhang
jedoch nicht bestätigt werden.
Als Folge dessen
ist eine andere Theorie notwendig,
um den inversen Faraday Effekt
in einem Bandisolator korrekt zu beschreiben.


\section*{Abstract}
\begin{english}
This bachelor thesis deals with the possibility to describe
the inverse Faraday effect a band-insulator by Floquet theory.
Therefore a 2D model of a band-insulator was considered,
where the electrons are assumed as spinless,
and the Floquet matrix of the system
was deduced.
It turned out that it was necessary, for the
correct description of the system, to regard the
higher Fourier modes in the Floquet
matrix for lower frequencies, higher
electric amplitudes
and lower local energies. Futhermore the
avarage current for a one electron system was calculated.
This avarage current showed a linear relationship to the
frequency and a quadratic relationship
to the light source's electric amplitude.
These relations were not
confirmed for a two
electron system.
As a result another theory
is necessery to describe
the inverse Faraday effect
in a Band-insulator exactly.
\end{english}
