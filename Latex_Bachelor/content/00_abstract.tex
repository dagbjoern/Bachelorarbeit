\thispagestyle{plain}

\section*{Kurzfassung}
Die Bachelorarbeit zeigt eine Möglichkeit auf den inversen Faraday Effekt
in einem Bandisolator
mit der Floquet Theorie zu beschreiben.
Es wird ein 2D Modell aufgestellt und aus diesem
eine Floquet Matix aufgestellt.
Es zeigte sich, dass es nötig ist für geringe Frequenzen des und Lokale Energien in Isolator die Flouqet Matrix
zu vergrößern, um das System korrekt zu beschreiben.
Desweiteren könnte in dem System, welches ein Elektron enthält, ein Strommittelwert
berechnet werden, der (die vorhergesagt )/(eine) Lineare Abhängigkeit
zu der Frequnez sowie eine quadratische Abhängigkeit
zu der elektrischen Amplitude der Lichtquelle besitzt.
Dies wurde ebenfalls für eine System mit zwei Elektronen überprüft,
jedoch konnte der zuvor gefundene Zusammenhang nicht bestätigt werden.
Somit ist eine andere Theorie notwendig, um den inversen Faraday Effekt
in einem Bandisolator genau zu beschreiben.


\section*{Abstract}
\begin{english}
The abstract is a short summary of the thesis in English, together with the German summary it has to fit on this page.

\end{english}
