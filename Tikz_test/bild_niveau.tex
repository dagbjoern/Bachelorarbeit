\documentclass{article}
\usepackage[pdftex,active,tightpage]{preview}
\usepackage{tikz}
\usepackage{mathpazo}

%------------------------------------------------------------------------------
%------------------------------ Sprache und Schrift: --------------------------
%------------------------------------------------------------------------------
\usepackage{fontspec}
\defaultfontfeatures{Ligatures=TeX}  % -- becomes en-dash etc.

% german language
\usepackage{polyglossia}
\setdefaultlanguage{german}

% for english abstract and english titles in the toc
\setotherlanguages{english}

% intelligent quotation marks, language and nesting sensitive
\usepackage[autostyle]{csquotes}

% microtypographical features, makes the text look nicer on the small scale
\usepackage{microtype}



\usepackage{amsmath}
\usepackage{amssymb}
\usepackage{mathtools}


% nice, small fracs for the text with \sfrac{}{}
\usepackage{xfrac}
% Enable Unicode-Math and follow the ISO-Standards for typesetting math
\usepackage[
  math-style=ISO,
  bold-style=ISO,
  sans-style=italic,
  nabla=upright,
  partial=upright,
]{unicode-math}
\setmathfont{Latin Modern Math}
\usetikzlibrary{decorations.pathreplacing}


\usepackage[
  locale=DE,
  separate-uncertainty=true,
  per-mode=symbol-or-fraction,
]{siunitx}
\sisetup{math-micro=\text{µ},text-micro=µ}

\PreviewEnvironment{tikzpicture}
% \newcounter{row}
% \newcounter{col}


\begin{document}
\begin{tikzpicture}
\draw (1,2) -- (-1,2)   node[left] {$\sqrt{a^2+4J^2}$};
\draw (1,1.2) -- (-1,1.2)   node[left] {$a$};
\draw (1,-1.2) -- (-1,-1.2) node[left] {$-a$};
\draw (1,-2) -- (-1,-2) node[left] {$-\sqrt{a^2+4J^2}$};
\draw [dashed] (-1,0) -- (1,0) node[left] {$0$};
\draw[decorate, decoration={brace}, yshift=-2ex]  (1,-0.9) -- node[right] {$\omega_{\mathrm{res}_1}$}  (1,-1.7);
\draw[decorate, decoration={brace}, yshift=-2ex]  (2,1.5) -- node[right] {$\omega_{\mathrm{res}_2}$}  (2,-1.7);
\draw[decorate, decoration={brace}, yshift=-2ex]  (3,2.3) -- node[right] {$\omega_{\mathrm{res}_3}$}  (3,-1.7);
%\node[below = 0.25cm] at (3,3) {$\vec{E}$};
\end{tikzpicture}
\end{document}
